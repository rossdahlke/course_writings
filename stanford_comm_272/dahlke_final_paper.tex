\documentclass[12pt,]{article}
\usepackage[left=1in,top=1in,right=1in,bottom=1in]{geometry}
\newcommand*{\authorfont}{\fontfamily{phv}\selectfont}
\usepackage[]{mathpazo}


  \usepackage[T1]{fontenc}
  \usepackage[utf8]{inputenc}




\usepackage{abstract}
\renewcommand{\abstractname}{}    % clear the title
\renewcommand{\absnamepos}{empty} % originally center

\renewenvironment{abstract}
 {{%
    \setlength{\leftmargin}{0mm}
    \setlength{\rightmargin}{\leftmargin}%
  }%
  \relax}
 {\endlist}

\makeatletter
\def\@maketitle{%
  \newpage
%  \null
%  \vskip 2em%
%  \begin{center}%
  \let \footnote \thanks
    {\fontsize{18}{20}\selectfont\raggedright  \setlength{\parindent}{0pt} \@title \par}%
}
%\fi
\makeatother




\setcounter{secnumdepth}{0}

\usepackage{longtable,booktabs}



\title{Retail investors and media psychology  }



\author{\Large Ross Dahlke\vspace{0.05in} \newline\normalsize\emph{}  }


\date{}

\usepackage{titlesec}

\titleformat*{\section}{\normalsize\bfseries}
\titleformat*{\subsection}{\normalsize\itshape}
\titleformat*{\subsubsection}{\normalsize\itshape}
\titleformat*{\paragraph}{\normalsize\itshape}
\titleformat*{\subparagraph}{\normalsize\itshape}





\newtheorem{hypothesis}{Hypothesis}
\usepackage{setspace}


% set default figure placement to htbp
\makeatletter
\def\fps@figure{htbp}
\makeatother

\usepackage{graphicx}

% move the hyperref stuff down here, after header-includes, to allow for - \usepackage{hyperref}

\makeatletter
\@ifpackageloaded{hyperref}{}{%
\ifxetex
  \PassOptionsToPackage{hyphens}{url}\usepackage[setpagesize=false, % page size defined by xetex
              unicode=false, % unicode breaks when used with xetex
              xetex]{hyperref}
\else
  \PassOptionsToPackage{hyphens}{url}\usepackage[draft,unicode=true]{hyperref}
\fi
}

\@ifpackageloaded{color}{
    \PassOptionsToPackage{usenames,dvipsnames}{color}
}{%
    \usepackage[usenames,dvipsnames]{color}
}
\makeatother
\hypersetup{breaklinks=true,
            bookmarks=true,
            pdfauthor={Ross Dahlke ()},
             pdfkeywords = {},  
            pdftitle={Retail investors and media psychology},
            colorlinks=true,
            citecolor=blue,
            urlcolor=blue,
            linkcolor=magenta,
            pdfborder={0 0 0}}
\urlstyle{same}  % don't use monospace font for urls

% Add an option for endnotes. -----


% add tightlist ----------
\providecommand{\tightlist}{%
\setlength{\itemsep}{0pt}\setlength{\parskip}{0pt}}

% add some other packages ----------

% \usepackage{multicol}
% This should regulate where figures float
% See: https://tex.stackexchange.com/questions/2275/keeping-tables-figures-close-to-where-they-are-mentioned
\usepackage[section]{placeins}


\begin{document}
	
% \pagenumbering{arabic}% resets `page` counter to 1 
%
% \maketitle

{% \usefont{T1}{pnc}{m}{n}
\setlength{\parindent}{0pt}
\thispagestyle{plain}
{\fontsize{18}{20}\selectfont\raggedright 
\maketitle  % title \par  

}

{
   \vskip 13.5pt\relax \normalsize\fontsize{11}{12} 
\textbf{\authorfont Ross Dahlke} \hskip 15pt \emph{\small }   

}

}








\begin{abstract}

    \hbox{\vrule height .2pt width 39.14pc}

    \vskip 8.5pt % \small 

\noindent Recent events surrounding GameStop stock, the trading platform
Robinhood, and the investor forum r/WallStreetBets have shown the
significance of individual investors on financial markets. Business
scholars have extensively studied how media and information can impact
financial markets. However, there remains a gap in the literature of how
individual investors seek out, process, and act on media. Media
psychologists can help to fill this research gap.


    \hbox{\vrule height .2pt width 39.14pc}


\end{abstract}


\vskip -8.5pt


 % removetitleabstract

\noindent \doublespacing 

\begin{longtable}[]{@{}lll@{}}
\toprule
Method & koRpus & stringi\tabularnewline
\midrule
\endhead
Word count & 2566 & 2587\tabularnewline
Character count & 17389 & 17388\tabularnewline
Sentence count & 143 & Not available\tabularnewline
Reading time & 12.8 minutes & 12.9 minutes\tabularnewline
\bottomrule
\end{longtable}

\hypertarget{introduction}{%
\section{Introduction}\label{introduction}}

Financial markets were recently upended by massive short squeezes of a
cadre of stocks including GameStop, AMC Entertainment Holdings,
BlackBerry, and Nokia. The recent events saw historic fluctuations in
the prices of these stocks. Individual investors made tens-of-millions
of dollars, often with leveraged positions. What is particularly unique
about these events is that they were triggered by retail investors.
These retail investors, bolstered by financial stimuli and an abundance
of free time due to the COVID-19 pandemic, organized collective behavior
actions on an internet Reddit subforum r/WallStreetBets.

Media commentators quickly offered various perspectives on the events.
Is this class warfare? A populist movement? Just some bored, lonely
young men trying to make money? There is a deep pool of academic
literature that explores the psychological motivations of retail
investors and personal finance more broadly that can shed light on these
investors' actions. For example: how gender, age, marital status,
income, and other demographics impact trading behaviors (Kannadhasan
2015); how heuristics are used to build trust between individual and
financial advisers (Monti et al. 2014); or how the psychological
dimensions of financial anxiety, optimism, financial security,
deliberative thinking, interest in financial issues, and needs for
precautionary savings all drive individual financial decisions (Talwar
et al. 2021). In addition, there is a robust and powerful set of
research on the impacts of media on financial markets. However, I am
interested in how media impact financial markets in-so-much-as how the
media effects, and is in turn affected by, individual investors.

More broadly, my research investigates how individuals enter and
participate in complex systems. In this case, the totality of the
financial markets are beyond comprehension. Investors, particularly
small, retail traders, are unlikely to ever fully understand the complex
system of global finance. Yet, they participate in the system anyway.
And as is increasingly common, participants are coordinating activity in
order to alter the systems that they are in. As it pertains to retail
investors, I ultimately want to know both \emph{why} they participate in
the markets, and more importantly, \emph{how} they participate in the
financial markets. What is the media ecology that individual investors
are exposed to or seek out? How do novice traders come to trust the
financial investment advice of various media outlets, communities, or
individuals? What do people do after they lose their life savings in
GameStop stock? How much time do retail investors devote to researching
financial securities?

\hypertarget{covid-19-and-retail-traders}{%
\subsection{COVID-19 and Retail
Traders}\label{covid-19-and-retail-traders}}

The COVID-19 global pandemic has spurred a rise in retail investors that
rivals the boom of retail investors during the internet bubble after the
rise of discount stock brokerages. At the center of the rise in retail
investors is Robinhood. Robinhood is a smartphone app and stock trading
platform that offers free trading of stocks and other securities. While
other trading platforms have since adopted free trading, Robinhood
created and maintained a significant following by being the first to
offer free trading, its gamified user interface, and a target
demographic of young, new investors. With Robinhood easing barriers of
entry for new investors, financial stimuli from the government due to
the pandemic, and newly-found swathes of free-time, these retail
investors grew in size and influence during the pandemic.

Retail investors now routinely account for 20\% of stock market activity
(Pagano et al. 2021). These retail investors on Robinhood significantly
impact financial markets, particularly during COVID (Pagano, Sedunov,
and Velthuis 2020). Users of Robinhood increased the amount of money
they invested on the platform during the pandemic (Welch 2020).
Robinhood traders primarily engage in both momentum and contrarian
strategies where they invest in stocks that have already demonstrated
rising price momentum and ``buying the dip'' where they buy stocks that
have recently fallen in price, respectively (Pagano, Sedunov, and
Velthuis 2020). In other words, Robinhood investors did not panic during
broad market turmoil during COVID-19 and often used price drops as
buying opportunities. However, in aggregate they do not produce an
alpha, or ``beat the market'' by achieving larger-than-average returns
(Pagano et al. 2021). Their trades ultimately produce noise in financial
markets (Pagano et al. 2021). If all Robinhood investors do is create
``noise'' in the markets, then the noisiest they've been is when
GameStop stock rose dramatically in early 2021.

\hypertarget{gamestop}{%
\subsection{GameStop}\label{gamestop}}

In January and February 2021, a handful of stocks experienced an extreme
``short-squeeze'' which effectively sky-rocketed the prices of these
stocks. These fluctuations were caused by decentralized retail investors
acting in concert (Lyócsa, Baumöhl, and Vŷrost 2021). The largest
movement was seen in GameStop which took most of the attention of the
saga. Many investors, both individual and institutional, gained and lost
significant sums of money. For a full recapping of the sequence of
events, I recommend Lopatto (2021) and Regnier (2021).

The GameStop saga provides support for the principle that market forces
tend to make markets fair, ``where fairness is defined as investors
`getting what they pay for' rather than investors `beating the
market'\,'' due to the apparent misvaluation of GameStop and other
stocks by institutional investors (Macey 2021). On the other hand, the
events also cast doubt as to whether the SEC is achieving its stated
goal of maintaining fair, orderly, and efficient markets" (Macey 2021).

The retail investors who caused the short squeeze organized on a
subforum of the popular website Reddit called ``Wall Street Bets,'' or
r/wallstreetbets (Lyócsa, Baumöhl, and Vŷrost 2021). As a result, the
GameStop saga is seen as battle between Wall Street and small, Robinhood
traders who are upset at Wall Street. The members of r/wallstreetbets
maintained a sense of unity and purpose throughout the events (Muzio
2021). Regardless of whether the organized behavior of r/wallstreetbets
users is a political movement, hopeful acts of ``sticking it to
hedgefunds'' or something else, they have demonstrated their power to
move financial markets (Muzio 2021). However, retail investors as whole
both bought and ``shorted''--or bet that GameStop stock was going to go
down--indicating that ``the Gamestop frenzy was not a pure digital
protest against Wall Street but speculative trading by a group of retail
investors, in line with their prior high-risk trading behavior'' (Hasso
et al. 2021).

\hypertarget{rwallstreetbets}{%
\subsection{r/WallStreetBets}\label{rwallstreetbets}}

Wall Street Bets is a subreddit where investors talk about stocks. The
group is known for their risky ``YOLO'' (You Only Live Once) trades
(Muzio 2021). There are a particular language and meme culture that
permeates the group and potentially helps to create their culture of
``degeneracy'' (Boylston et al. 2021). For example, they celebrate
people with ``diamond hands'' as those who are prepared to hold their
stocks for a long time and decry ``paper hands,'' those who miss out on
profits by selling a stock too early (Wallstreetbets News 2020). They
maintain they are not ``dumb,'' but they are ``retarded'' or
``autists.'' Although these words are derogatory, they use it as a way
to self-deprecate and build unity.

\hypertarget{information-and-media}{%
\section{Information and Media}\label{information-and-media}}

Scholars have been interested in the role of information on financial
markets. Business researchers largely operate under the efficient market
hypothesis where markets reflect accurate valuations based on available
information. As such, they are concerned about whether information being
spread about companies is timely or ``stale'' and the effects of media
attention on stocks, including newspaper, television, digital news, and
social media.

\hypertarget{newspapers}{%
\subsection{Newspapers}\label{newspapers}}

Even though one may think of the stock market as reacting to news
instantaneously, media attention given to individual stocks via
newspapers impact the pricing of the stock. The causal link between
newspapers impacting trading prices and volumes was established by
leveraging newspaper strikes and observing the subsequent impact on
financial markets (Peress 2014). This effect persists for many days
after publication, and the impact is prolonged during a recession
compared to an expansion (Antweiler and Frank 2006). However, not all
coverage is good coverage. Stocks with little or no attention in
newspapers earn higher returns than high-media-attention stocks (Fang
and Peress 2009).

Stock market investors respond differently to new versus stale news--as
defined by textual similarity of sequential newspaper stories. The
prices of stocks respond less to stale news, but there is a reversal in
the movement of a stock's price after stale news, suggesting that
initial movements are overreactions (Tetlock 2011). Further,
quantitative information is easier for news readers to process compared
to qualitative information (Engelberg 2008).

In addition, it is not just substantive news about a stock that impacts
performance, stock recommendations in the \emph{Wall Street Journal's}
``Dartboard'' column predicted abnormal positive returns of 4\% and
double the trading volume over the two days after the column recommended
a stock (Barber and Loeffler 1993). These market abnormalities are
largely driven by naive investors and are ultimately noise in the
financial markets. Price movements from the column were reversed within
15 days, and investors following the ``expert'' recommendations of the
column lost 3.8\% on a risk-adjusted basis in the six months following
the recommendation (Liang 1999). These findings show that media content
has a clear relationship with asset prices and should not be seen as a
``sideshow,'' are consistent with the theoretical models of retail
investors acting as noise in the financial ecosystem, and is
inconsistent with the theory that media can serve as a proxy for new
financial information that informs the efficient market on fundamental
asset valuations (Tetlock 2007).

\hypertarget{television}{%
\subsection{Television}\label{television}}

Similar to newspapers, the financial markets react to media content on
television, often in real-time. When the TV program ``The Morning Call''
on \emph{CNBC} reports analyst views on individuals stocks, the market
reacts within seconds, the price is fully incorporated within one
minute, and trading intensity in the stock doubles within that first
minute (Busse and Clifton Green 2002). ``Mad Money with Jim Cramer'' is
a popular \emph{CNBC} television show where the titular host, Jim
Cramer, makes a bevy of buy and sell recommendations on stocks. There
are significant price movements for stocks that Cramer recommends to buy
and sell (Bolster, Trahan, and Venkateswaran 2012; Engelberg,
Sasseville, and Williams 2012; Karniouchina, Moore, and Cooney 2009).
The effects on the movement of stocks given a ``buy'' rating from Cramer
are quickly reversed, but the effect persists longer for ``sell''
recommendations (Bolster, Trahan, and Venkateswaran 2012). The viewing
ratings of the show even predict the strength of the price movements
(Engelberg, Sasseville, and Williams 2012). Further, traditional
advertising variables, such as message length, recency-primacy effects,
information clutter, and source credibility all influence the size of
the market reaction to a ``buy'' recommendation from Cramer
(Karniouchina, Moore, and Cooney 2009). These findings suggest that
content consumers process media from financial pundits similarly to
other types of advertising. Ultimately, these effects show that the
investors taking actions from Cramer's recommendations are uninformed
traders and not experienced investors receiving any new information
(Keasler and McNeil 2010).

\hypertarget{social-media}{%
\subsection{Social Media}\label{social-media}}

As evidenced by the recent GameStop and r/wallstreetbets saga, social
media is playing an increasing role in the way that individual investors
collect, share, and process information that results in investment
decisions. The literature on social media and the stock market fall into
three broad categories: social media as a predictor of stock movements,
crowd-sourced financial recommendations, and behavioral contagion/
herding.

First, social media has been used to try to predict fluctuations in
individual stocks. The number of posts on TheRagingBull.com, one of the
original stock message boards, in 2001 coincided with days with
abnormally high trading volume, but they were not predictive of market
pricing or volume after for controlling for industry-adjusted returns or
normal trading volume (Tumarkin and Whitelaw 2001). This result suggests
that some other exogenous events are confounders that drive both social
media engagement and stock market volume. However, individual tweets
just before a firm's quarterly earnings announcement is a predictor of
the earnings report and subsequent price action (Mohanram 2018). News
diffusion on Twitter leads to lower bid-ask spread and price pressure
during a news day, but the effect is reversed the next day (Ye 2016).
Twitter, like traditional media, spreads stale news, albeit at a higher
speed than traditional media (Ye 2016).

Other new media, including Yahoo! stock message boards (Das and Chen
2007) and Google Search Volume (Gao 2011) are all predictive of stock
movements. Even disagreements on StockTwits, a Twitter-like stock-based
social media, predict price changes by affecting trading done by
uninformed investors and facilitates trading by informed traders who are
taking actions aimed at changing corporate policies (Cookson, Fos, and
Niessner 2021).

Next, social media have been studied as a way to crowdsource financial
advice. For example, Seeking Alpha is a website where non-professional
investors can write articles analyzing a stock. Seeking Alpha articles
are predictors of stock market movements (Campbell, DeAngelis, and Moon
2019; Chen et al. 2014; Farrell et al. 2018). Estimize is an open
platform that solicits forecasts from contributors. These crowdsourced
forecasts are incrementally useful in forecasting earnings, and the
higher the volume of forecasts, the more accurate the predictions are
(Jame et al. 2016). SumZero.com is a private social networking site that
facilitates interaction and information exchange among professional
investors. Recommendations offered on this website generate significant
returns in the financial markets (Crawford et al. 2018). Long-term
returns are particularly strong for ``contrarian'' buy recommendations
(Crawford et al. 2018).

Finally, social media can create ``herding'' by which there is
behavioral contagion that flows through networks of investors via their
connections on social media. Social network links increase disposition
effects (Heimer 2016), which is the tendency to sell ``winning'' assets
and hold onto ``losing'' assets (Shefrin and Statman 1985). Herding is
particularly prominent on Robinhood (Barber et al. 2021). And the
GameStop saga provides evidence for the idea of behavioral contagion
among r/wallstreetbets users (Semenova and Winkler 2021). On StockTwits
there is even evidence of echo chambers where users selectively expose
themselves to other users who share positive or negative sentiments on
specific stocks (Cookson, Engelberg, and Mullins 2020). The herding
behavior found in online investor communities also happens in offline
networks (Duflo and Saez 2002, 2003; Ivković and Weisbenner 2007;
Musciotto et al. 2018; Shiller and Pound 1989). Potentially, this
similarity between online and offline social behavioral contagion can
add to the digitization and social network literature.

\hypertarget{conclusion-future-research}{%
\section{Conclusion \& Future
Research}\label{conclusion-future-research}}

The relationship between media, information, and the stock market is
well-studied. But approaching the topic from a media psychology angle
provides new questions and avenues of study. It is unclear at what time
intervals individual investors make trading decisions. How long do
retail investors take before purchasing a stock? Do different types of
information from sources of varying credibility impact the scale of time
in making a decision? Taking a Lemkeian approach (2000) to understanding
the temporal dimension of investment decisions could help to answer
these questions.

Another area of future research is how the design of media and trading
platforms impacts financial decisions. There has been a lot of
speculation on how Robinhood's gamified interface spurs more and riskier
trades. How do the design of financial platforms and financial media
alter trading behaviors?

There is a defined relationship between media impacting the markets, but
there is little research on the reverse: how do changing market dynamics
alter information-seeking behaviors? What do people do when the market
is having a good day? A bad day? How do people even go about getting
information to make a trading decision? How is trust built between
someone like Jim Cramer or a random Reddit user and investors?

Finally, digital technologies have fragmented the media experiences of
users. Instead of sitting down at one's computer and intensely trading
stocks, many users now pull up a trading app on their phone and casually
trade. Are most people using the Robinhood app for hours at a time?
Unlikely, instead, it is more likely that there are short bursts of
attention given to a trading app intertwined with other media. Can
interacting with other types of media impact how one trades? Does
trading behavior impact the other media that a trader consumes? All of
these questions remain unanswered.

Scholars of financial markets have intensely studied the psychology of
investing and personal finance. They have also studied relationships
between media and financial markets. But all of their research questions
revolve around studying the market as the outcome. Media psychologists
can add to this area of research by studying individual behavior as the
dependent variable.

\hypertarget{references}{%
\section*{References}\label{references}}
\addcontentsline{toc}{section}{References}

\hypertarget{refs}{}
\leavevmode\hypertarget{ref-antweiler2006}{}%
Antweiler, Werner, and Murray Z. Frank. 2006. ``Do Us Stock Markets
Typically Overreact to Corporate News Stories.'' In.

\leavevmode\hypertarget{ref-barber2021}{}%
Barber, Brad M., Xing Huang, Terrance Odean, and Christopher Schwarz.
2021. ``Attention Induced Trading and Returns: Evidence from Robinhood
Users.'' \emph{Available at SSRN: Https://Ssrn.com/Abstract=3715077 or
Http://Dx.doi.org/10.2139/Ssrn.3715077}.

\leavevmode\hypertarget{ref-barber1993}{}%
Barber, Brad M., and Douglas Loeffler. 1993. ``The "Dartboard" Column:
Second-Hand Information and Price Pressure.'' \emph{The Journal of
Financial and Quantitative Analysis} 28 (2): 273--84.
\url{http://www.jstor.org/stable/2331290}.

\leavevmode\hypertarget{ref-bolster2012}{}%
Bolster, Paul, Emery Trahan, and Anand Venkateswaran. 2012. ``How Mad Is
Mad Money? Jim Cramer as a Stock Picker and Portfolio Manager.''
\emph{The Journal of Investing} 21 (2): 27--39.
\url{https://doi.org/10.3905/joi.2012.21.2.027}.

\leavevmode\hypertarget{ref-boylston2021}{}%
Boylston, Christian, Beatriz Palacios, Plamen Tassev, and Amy Bruckman.
2021. ``WallStreetBets: Positions or Ban.''
\url{http://arxiv.org/abs/2101.12110}.

\leavevmode\hypertarget{ref-busse2002}{}%
Busse, Jeffrey A., and T. Clifton Green. 2002. ``Market Efficiency in
Real Time.'' \emph{Journal of Financial Economics} 65 (3): 415--37.
\url{https://doi.org/https://doi.org/10.1016/S0304-405X(02)00148-4}.

\leavevmode\hypertarget{ref-campbell2019}{}%
Campbell, John L., Matthew D. DeAngelis, and James R. Moon. 2019. ``Skin
in the game: personal stock holdings and investors' response to stock
analysis on social media.'' \emph{Review of Accounting Studies} 24 (3):
731--79. \url{https://doi.org/10.1007/s11142-019-09498-}.

\leavevmode\hypertarget{ref-chen2014}{}%
Chen, Hailiang, Prabuddha De, Yu (Jeffrey) Hu, and Byoung-Hyoun Hwang.
2014. ``Wisdom of Crowds: The Value of Stock Opinions Transmitted
Through Social Media.'' \emph{The Review of Financial Studies} 27 (5):
1367--1403. \url{https://doi.org/10.1093/rfs/hhu001}.

\leavevmode\hypertarget{ref-cookson2020}{}%
Cookson, J. Anthony, Joseph E. Engelberg, and William Mullins. 2020.
``Echo Chambers.'' \emph{SocArXiv1}, June.
\url{https://doi.org/doi:10.31235/osf.io/n2q9h}.

\leavevmode\hypertarget{ref-cookson2021}{}%
Cookson, J. Anthony, Vyacheslav Fos, and Marina Niessner. 2021. ``Does
Disagreement Facilitate Informed Trading? Evidence from Activist
Investors.'' \emph{Available at SSRN: Https://Ssrn.com/Abstract=3765092
or Http://Dx.doi.org/10.2139/Ssrn.3765092}.

\leavevmode\hypertarget{ref-crawford2018}{}%
Crawford, Steven, Wesley Gray, Bryan R. Johnson, and Richard A. Price.
2018. ``What Motivates Buy-Side Analysts to Share Recommendations
Online?'' \emph{Management Science} 64 (6): 2574--89.
\url{https://doi.org/10.287/mnsc.2017.2749}.

\leavevmode\hypertarget{ref-das2007}{}%
Das, Sanjiv R., and Mike Y. Chen. 2007. ``Yahoo! for Amazon: Sentiment
Extraction from Small Talk on the Web.'' \emph{Management Science} 53
(9): 1375--88. \url{https://doi.org/10.1287/mnsc.1070.0704}.

\leavevmode\hypertarget{ref-duflo2002}{}%
Duflo, Esther, and Emmanuel Saez. 2002. ``Participation and investment
decisions in a retirement plan: the influence of colleagues' choices.''
\emph{Journal of Public Economics} 85 (1): 121--48.
\url{https://ideas.repec.org/a/eee/pubeco/v85y2002i1p121-148.html}.

\leavevmode\hypertarget{ref-duflo2003}{}%
---------. 2003. ``The Role of Information and Social Interactions in
Retirement Plan Decisions: Evidence from a Randomized Experiment*.''
\emph{The Quarterly Journal of Economics} 118 (3): 815--42.
\url{https://doi.org/10.1162/00335530360698432}.

\leavevmode\hypertarget{ref-engelberg2008}{}%
Engelberg, Joseph. 2008. ``Costly Information Processing: Evidence from
Earnings Announcements.'' \emph{AFA 2009 San Francisco Meetings Paper},
January.
\href{https://doi.org/http://dx.doi.org/10.2139/ssrn.1107998\%20}{https://doi.org/http://dx.doi.org/10.2139/ssrn.1107998}.

\leavevmode\hypertarget{ref-engelberg2012}{}%
Engelberg, Joseph, Caroline Sasseville, and Jared Williams. 2012.
``Market Madness? The Case of "Mad Money".'' \emph{Management Science}
58 (2): 351--64. \url{http://www.jstor.org/stable/41406393}.

\leavevmode\hypertarget{ref-fang2009}{}%
Fang, Lily, and Joel Peress. 2009. ``Media Coverage and the
Cross-Section of Stock Returns.'' \emph{The Journal of Finance} 64 (5):
2023--52.
\url{https://doi.org/https://doi.org/10.1111/j.1540-6261.2009.01493.x}.

\leavevmode\hypertarget{ref-farrell2018}{}%
Farrell, Michael, T. Clifton Green, Russell Jame, and Stanimir Markov.
2018. ``The Democratization of Investment Research and the
Informativeness of Retail Investor Trading.'' \emph{Available at SSRN:
Https://Ssrn.com/Abstract=3222841 or
Http://Dx.doi.org/10.2139/Ssrn.3222841}.

\leavevmode\hypertarget{ref-da2011}{}%
Gao, Zhi Da; Joseph Engelberg; Pengjie. 2011. ``In Search of
Attention.'' \emph{The Journal of Finance} 66 (5): 1461--99.
\url{https://doi.org/https://doi.org/10.1111/j.1540-6261.2011.01679.x}.

\leavevmode\hypertarget{ref-hasso2021}{}%
Hasso, Tim, Daniel Müller, Matthias Pelster, and Sonja Warkulat. 2021.
``Who Participated in the Gamestop Frenzy?'' \emph{TAF Working Paper No.
58}, February.

\leavevmode\hypertarget{ref-heimer2016}{}%
Heimer, Rawley Z. 2016. ``Peer Pressure: Social Interaction and the
Disposition Effect.'' \emph{The Review of Financial Studies} 29 (11):
3177--3209. \url{https://doi.org/10.1093/rfs/hhw063}.

\leavevmode\hypertarget{ref-ivkovic2007}{}%
Ivković, Zoran, and Scott Weisbenner. 2007. ``Information Diffusion
Effects in Individual Investors' Common Stock Purchases: Covet Thy
Neighbors' Investment Choices.'' \emph{The Review of Financial Studies}
20 (4): 1327--57. \url{https://doi.org/10.1093/revfin/hhm009}.

\leavevmode\hypertarget{ref-jame2016}{}%
Jame, Russell, Rick Johnston, Stanimir Markov, and Michael C. Wolfe.
2016. ``The Value of Crowdsourced Earnings Forecasts.'' \emph{Journal of
Accounting Research} 54 (4): 1077--1110.
\url{https://doi.org/https://doi.org/10.1111/1475-679X.12121}.

\leavevmode\hypertarget{ref-kannadhasan2015}{}%
Kannadhasan, M. 2015. ``Retail Investors' Financial Risk Tolerance and
Their Risk-Taking Behaviour: The Role of Demographics as Differentiating
and Classifying Factors.'' \emph{IIMB Management Review} 27 (3):
175--84.
\url{https://doi.org/https://doi.org/10.1016/j.iimb.2015.06.004}.

\leavevmode\hypertarget{ref-karniouchina2009}{}%
Karniouchina, Ekaterina V., William L. Moore, and Kevin J. Cooney. 2009.
``Impact of "Mad Money" Stock Recommendations: Merging Financial and
Marketing Perspectives.'' \emph{Journal of Marketing} 73 (6): 244--66.
\url{http://www.jstor.org/stable/20619072}.

\leavevmode\hypertarget{ref-keasler2010}{}%
Keasler, Terrill, and Chris McNeil. 2010. ``Mad Money stock
recommendations: market reaction and performance.'' \emph{Journal of
Economics and Finance} 34 (1): 1--22.
\url{https://doi.org/10.1007/s12197-008-9033-7}.

\leavevmode\hypertarget{ref-lemke}{}%
Lemke, Jay L. 2000. ``Across the Scales of Time: Artifacts, Activities,
and Meanings in Ecosocial Systems.'' \emph{Mind, Culture, and Activity}
7 (4): 273--90. \url{https://doi.org/10.1207/S15327884MCA0704/_03}.

\leavevmode\hypertarget{ref-liang1999}{}%
Liang, Bing. 1999. ``Price Pressure: Evidence from the
\&quot;Dartboard\&quot; Column.'' \emph{The Journal of Business} 72 (1):
119--34. \url{https://doi.org/10.1086/209604}.

\leavevmode\hypertarget{ref-lopatto2021}{}%
Lopatto, Elizabeth. 2021. ``How R/Wallstreetbets Gamed the Stock of
Gamestop.'' \emph{The Verge}, January.
\url{https://www.theverge.com/22251427/reddit-gamestop-stock-short-wallstreetbets-robinhood-wall-street}.

\leavevmode\hypertarget{ref-lyocsa2021}{}%
Lyócsa, Štefan, Eduard Baumöhl, and Tomáš Vŷrost. 2021. ``YOLO Trading:
Riding the Herd During the Gamestop Episode.'' \emph{ZBW - Leibniz
Information Centre for Economics, Working Paper}.
\url{http://hdl.handle.net/10419/230679}.

\leavevmode\hypertarget{ref-macey2021}{}%
Macey, Jonathan R. 2021. ``Securities Regulation as Class Warfare.''
\emph{Columbia Business Law Review, Forthcoming}, February.
\url{https://doi.org/http://dx.doi.org/10.2139/ssrn.3789706}.

\leavevmode\hypertarget{ref-bartov2018}{}%
Mohanram, Eli Bartov; Lucile Faurel; Partha S. 2018. ``Can Twitter Help
Predict Firm-Level Earnings and Stock Returns?'' \emph{The Accounting
Review} 93 (3): 25--57.
\url{https://doi.org/https://doi.org/10.2308/accr-51865}.

\leavevmode\hypertarget{ref-monti2014}{}%
Monti, Marco, Vittorio Pelligra, Laura Martignon, and Nathan Berg. 2014.
``Retail Investors and Financial Advisors: New Evidence on Trust and
Advice Taking Heuristics.'' \emph{Journal of Business Research} 67 (8):
1749--57.
\url{https://doi.org/https://doi.org/10.1016/j.jbusres.2014.02.022}.

\leavevmode\hypertarget{ref-musciotto2018}{}%
Musciotto, Federico, Luca Marotta, Jyrki Piilo, and Rosario N. Mantegna.
2018. ``Long-Term Ecology of Investors in a Financial Market.''
\emph{Palgrave Communication} 4 (1): 92.

\leavevmode\hypertarget{ref-dimuzio2021}{}%
Muzio, Tim Di. 2021. ``GameStop Capitalism. Wall Street Vs. The Reddit
Rally (Part 1).'' \emph{University of Wollongong Working Paper}, 1--13.
\url{http://bnarchives.yorku.ca/673/}.

\leavevmode\hypertarget{ref-eaton2021}{}%
Pagano, Gregory W., Clifton T. Green, Brian Roseman, and Yanbin Wu.
2021. ``Zero-Commission Individual Investors, High Frequency Traders,
and Stock Market Quality.'' \emph{Available at SSRN:
Https://Ssrn.com/Abstract=3776874 or
Http://Dx.doi.org/10.2139/Ssrn.3776874}, January.
\url{https://doi.org/http://dx.doi.org/10.2139/ssrn.3776874}.

\leavevmode\hypertarget{ref-pagano2020}{}%
Pagano, Michael S., John Sedunov, and Raisa Velthuis. 2020. ``How Did
Retail Investors Respond to the Covid-19 Pandemic? The Effect of
Robinhood Brokerage Customers on Market Quality.'' \emph{Finance
Research Letters, Forthcoming}, November.
\href{https://doi.org/http://dx.doi.org/10.2139/ssrn.3703815\%20}{https://doi.org/http://dx.doi.org/10.2139/ssrn.3703815}.

\leavevmode\hypertarget{ref-peress2014}{}%
Peress, Joel. 2014. ``The Media and the Diffusion of Information in
Financial Markets: Evidence from Newspaper Strikes.'' \emph{The Journal
of Finance} 69 (5): 2007--43.
\url{https://doi.org/https://doi.org/10.1111/jofi.12179}.

\leavevmode\hypertarget{ref-regnier2021}{}%
Regnier, Pat. 2021. ``Stonks Are Bonkers, and Other Lessons from the
Reddit Rebellion.'' \emph{Bloomberg.com}, February.
\url{https://www.bloomberg.com/news/features/2021-02-04/gamestop-gme-how-wallstreetbets-and-robinhood-created-bonkers-stock-market}.

\leavevmode\hypertarget{ref-semenova2021}{}%
Semenova, Valentina, and Julian Winkler. 2021. ``Reddit's Self-Organized
Bull Runs.'' \emph{Available at
Https://Mpra.ub.uni-Muenchen.de/105630/}.

\leavevmode\hypertarget{ref-shefrin1985}{}%
Shefrin, Hersh, and Meir Statman. 1985. ``The Disposition to Sell
Winners Too Early and Ride Losers Too Long: Theory and Evidence.''
\emph{The Journal of Finance} 40 (3): 777--90.
\url{https://doi.org/https://doi.org/10.1111/j.1540-6261.1985.tb05002.x}.

\leavevmode\hypertarget{ref-shiller1989}{}%
Shiller, 021Robert J., and John Pound. 1989. ``Survey evidence on
diffusion of interest and information among investors.'' \emph{Journal
of Economic Behavior \& Organization} 12 (1): 47--66.
\url{https://ideas.repec.org/a/eee/jeborg/v12y1989i1p47-66.html}.

\leavevmode\hypertarget{ref-talwar2021}{}%
Talwar, Manish, Shalini Talwar, Puneet Kaur, Naliniprava Tripathy, and
Amandeep Dhir. 2021. ``Has Financial Attitude Impacted the Trading
Activity of Retail Investors During the Covid-19 Pandemic?''
\emph{Journal of Retailing and Consumer Services} 58: 102341.
\url{https://doi.org/https://doi.org/10.1016/j.jretconser.2020.102341}.

\leavevmode\hypertarget{ref-tetlock2007}{}%
Tetlock, Paul C. 2007. ``Giving Content to Investor Sentiment: The Role
of Media in the Stock Market.'' \emph{The Journal of Finance} 62 (3):
1139--68.
\url{https://doi.org/https://doi.org/10.1111/j.1540-6261.2007.01232.x}.

\leavevmode\hypertarget{ref-tetlock2011}{}%
---------. 2011. ``All the News That's Fit to Reprint: Do Investors
React to Stale Information?'' \emph{The Review of Financial Studies} 24
(5): 1481--1512. \url{https://doi.org/10.1093/rfs/hhq141}.

\leavevmode\hypertarget{ref-tumarkin2001}{}%
Tumarkin, Robert, and Robert F. Whitelaw. 2001. ``News or Noise?
Internet Postings and Stock Prices.'' \emph{Financial Analysts Journal}
57 (3): 41--51. \url{https://doi.org/10.2469/faj.v57.n3.2449}.

\leavevmode\hypertarget{ref-wsbshop}{}%
Wallstreetbets News. 2020. ``Dissecting the Unique Lingo and Terminolgy
Used in the Subreddit R/Wallstreetbets.''
\url{\%22https://www.wallstreetbets.shop/blogs/news/dissecting-the-unique-lingo-and-terminology-used-in-the-subreddit-r-wallstreetbets\%22}.

\leavevmode\hypertarget{ref-welch2020}{}%
Welch, Ivo. 2020. ``The Wisdom of the Robinhood Crowd.'' \emph{NBER
Working Paper Series}, September.
\url{http://www.nber.org/papers/w27866}.

\leavevmode\hypertarget{ref-chawla2016}{}%
Ye, Nitesh Chawla; Zhi Da; Jian Xu; Mao. 2016. ``Information Diffusion
on Social Media: Does It Affect Trading, Return, and Liquidity.''
\emph{Available at SSRN: Https://Ssrn.com/Abstract=2935138 or
Http://Dx.doi.org/10.2139/Ssrn.2935138}.





\newpage
\singlespacing 
\end{document}
