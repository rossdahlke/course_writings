\documentclass[12pt,]{article}
\usepackage[left=1in,top=1in,right=1in,bottom=1in]{geometry}
\newcommand*{\authorfont}{\fontfamily{phv}\selectfont}
\usepackage[]{mathpazo}


  \usepackage[T1]{fontenc}
  \usepackage[utf8]{inputenc}




\usepackage{abstract}
\renewcommand{\abstractname}{}    % clear the title
\renewcommand{\absnamepos}{empty} % originally center

\renewenvironment{abstract}
 {{%
    \setlength{\leftmargin}{0mm}
    \setlength{\rightmargin}{\leftmargin}%
  }%
  \relax}
 {\endlist}

\makeatletter
\def\@maketitle{%
  \newpage
%  \null
%  \vskip 2em%
%  \begin{center}%
  \let \footnote \thanks
    {\fontsize{18}{20}\selectfont\raggedright  \setlength{\parindent}{0pt} \@title \par}%
}
%\fi
\makeatother




\setcounter{secnumdepth}{0}

\usepackage{longtable,booktabs}



\title{Retail investors and media psychology  }



\author{\Large Ross Dahlke\vspace{0.05in} \newline\normalsize\emph{}  }


\date{}

\usepackage{titlesec}

\titleformat*{\section}{\normalsize\bfseries}
\titleformat*{\subsection}{\normalsize\itshape}
\titleformat*{\subsubsection}{\normalsize\itshape}
\titleformat*{\paragraph}{\normalsize\itshape}
\titleformat*{\subparagraph}{\normalsize\itshape}





\newtheorem{hypothesis}{Hypothesis}
\usepackage{setspace}


% set default figure placement to htbp
\makeatletter
\def\fps@figure{htbp}
\makeatother

\usepackage{graphicx}

% move the hyperref stuff down here, after header-includes, to allow for - \usepackage{hyperref}

\makeatletter
\@ifpackageloaded{hyperref}{}{%
\ifxetex
  \PassOptionsToPackage{hyphens}{url}\usepackage[setpagesize=false, % page size defined by xetex
              unicode=false, % unicode breaks when used with xetex
              xetex]{hyperref}
\else
  \PassOptionsToPackage{hyphens}{url}\usepackage[draft,unicode=true]{hyperref}
\fi
}

\@ifpackageloaded{color}{
    \PassOptionsToPackage{usenames,dvipsnames}{color}
}{%
    \usepackage[usenames,dvipsnames]{color}
}
\makeatother
\hypersetup{breaklinks=true,
            bookmarks=true,
            pdfauthor={Ross Dahlke ()},
             pdfkeywords = {},  
            pdftitle={Retail investors and media psychology},
            colorlinks=true,
            citecolor=blue,
            urlcolor=blue,
            linkcolor=magenta,
            pdfborder={0 0 0}}
\urlstyle{same}  % don't use monospace font for urls

% Add an option for endnotes. -----


% add tightlist ----------
\providecommand{\tightlist}{%
\setlength{\itemsep}{0pt}\setlength{\parskip}{0pt}}

% add some other packages ----------

% \usepackage{multicol}
% This should regulate where figures float
% See: https://tex.stackexchange.com/questions/2275/keeping-tables-figures-close-to-where-they-are-mentioned
\usepackage[section]{placeins}


\begin{document}
	
% \pagenumbering{arabic}% resets `page` counter to 1 
%
% \maketitle

{% \usefont{T1}{pnc}{m}{n}
\setlength{\parindent}{0pt}
\thispagestyle{plain}
{\fontsize{18}{20}\selectfont\raggedright 
\maketitle  % title \par  

}

{
   \vskip 13.5pt\relax \normalsize\fontsize{11}{12} 
\textbf{\authorfont Ross Dahlke} \hskip 15pt \emph{\small }   

}

}








\begin{abstract}

    \hbox{\vrule height .2pt width 39.14pc}

    \vskip 8.5pt % \small 

\noindent Recent events surrounding GameStop stock, the trading platform
Robinhood, and the investor forum r/WallStreetBets have shown how media
can drive massive fluctuations in global financial markets. Business
scholars have extensively studied how media and information contribute
to movements in stock prices. However, there remains a gap in the
literature of how individual investors seek out, process, and act on
media. Media psychologists can help to fill this research gap. Instead
of studying media's effect on financial markets, media psychologists
should study how media relate to individuals who are participating in
financial markets. In other words, media psychologists should study
people, not the markets, as the dependent variable.


    \hbox{\vrule height .2pt width 39.14pc}


\end{abstract}


\vskip -8.5pt


 % removetitleabstract

\noindent \doublespacing 

\begin{longtable}[]{@{}lll@{}}
\toprule
Method & koRpus & stringi\tabularnewline
\midrule
\endhead
Word count & 2963 & 2987\tabularnewline
Character count & 20129 & 20128\tabularnewline
Sentence count & 160 & Not available\tabularnewline
Reading time & 14.8 minutes & 14.9 minutes\tabularnewline
\bottomrule
\end{longtable}

\hypertarget{introduction}{%
\section{Introduction}\label{introduction}}

Financial markets were recently upended by historic price fluctuations
in a cadre of stocks including GameStop, AMC Entertainment Holdings,
BlackBerry, and Nokia. Individual investors made tens-of-millions of
dollars, often with leveraged positions. What is particularly unique
about these events is that they were triggered by individual investors.
These retail investors--bolstered by government-issued COVID-19 stimulus
checks, Robinhood, a free stock trading app, and an abundance of free
time due to the COVID-19 pandemic--organized collective behavior actions
on an internet Reddit subforum r/WallStreetBets.

Media commentators were quick to speculate on the meaning of the events,
often pointing to class warfare, new-age populism, or young, lonely men
being bored at home. However, there is a large body of academic
literature on psychological motivators of investing and personal
finance. For example, gender, age, marital status, income, and other
demographics impact trading behaviors (Kannadhasan, 2015). Or, that most
investment decisions can be boiled down to individuals' psychological
dimensions of financial anxiety, optimism, financial security,
deliberative thinking, interest in financial issues, and needs for
precautionary savings all drive individual financial decisions (Talwar
et al., 2021). However, how individual investors consume and influence
media remains an open question. While business scholars have established
the causal link between media and changes in stock prices, they miss the
step in the causal diagram of how individual investors seek out,
consume, and act on media.

First, I discuss relevant academic research to the recent GameStop saga,
specifically talking about retail traders using Robinhood during
COVID-19, academic perspectives on the GameStop price movements, and
r/WallStreetBets. Then, I summarize scholarship into how media affect
financial markets, including newspapers, television, and social media.
Finally, I show how gaps in our understanding of individual investors
and how they consume and act-on social media can be addressed by taking
a media psychology approach.

\hypertarget{covid-19-and-retail-traders}{%
\subsection{COVID-19 and Retail
Traders}\label{covid-19-and-retail-traders}}

The COVID-19 global pandemic has spurred a rise in retail investors that
rivals the boom of retail investors during the internet bubble. At the
center of the rise in retail investors is Robinhood. Robinhood is a
smartphone app and stock trading platform that offers free trading of
stocks and other securities. While other trading platforms have since
adopted free trading, Robinhood created and maintained a significant
following by being the first to offer free trading, gamification of its
user interface, and targeting young, new investors. With Robinhood
easing barriers of entry for new investors, financial stimulus checks
from the government due to the pandemic, and newly-found swathes of
free-time, these retail investors grew in size and influence during the
pandemic.

Retail investors now routinely account for 20\% of stock market activity
(Pagano et al., 2021). These retail investors on Robinhood significantly
impact financial markets by driving stock prices up or down,
particularly during COVID (Pagano et al., 2020). Users of Robinhood
increased the amount of money they invested on the platform during the
pandemic (Welch, 2020). Robinhood traders primarily engage in momentum
and contrarian strategies. For momentum trades, they invest in stocks
that have already demonstrated rising price momentum. They also
participate in contrarian strategies when they ``buy the dip'' where
they purchase stocks that have recently fallen in price, respectively
(Pagano et al., 2020). In other words, Robinhood investors did not panic
during broad market turmoil during COVID-19 and often used price drops
as buying opportunities. However, in aggregate they do not produce an
alpha, or ``beat the market'' by achieving larger-than-average returns
(Pagano et al., 2021). Their trades ultimately produce noise in
financial markets (Pagano et al., 2021). If all Robinhood investors do
is create ``noise'' in the markets, then the noisiest they've been is
when GameStop stock rose--and fell--dramatically in early 2021.

\hypertarget{gamestop}{%
\subsection{GameStop}\label{gamestop}}

In January and February 2021, a handful of stocks experienced an extreme
climb in valuation which effectively sky-rocketed the prices of these
stocks. These fluctuations were caused by decentralized retail investors
acting in concert (Lyócsa et al., 2021). The largest movement was seen
in GameStop which took most of the attention of the saga. Many
investors, both individual and institutional, gained and lost
significant sums of money. For a full recapping of the sequence of
events, I recommend Lopatto (2021) and Regnier (2021).

The GameStop saga provides support for the principle that market forces
tend to make markets fair, ``where fairness is defined as investors
`getting what they pay for' rather than investors `beating the
market'\,'' due to the apparent misvaluation of GameStop and other
stocks by institutional investors (Macey, 2021). On the other hand, the
events also cast doubt as to whether the SEC is achieving its stated
goal of maintaining ``fair, orderly, and efficient markets'' (Macey,
2021).

The retail investors who caused massive price increased organized on a
subforum of the popular website Reddit called ``Wall Street Bets,'' or
r/WallStreetBets (Lyócsa et al., 2021). As a result, the GameStop saga
is seen as a battle between Wall Street and small, Robinhood traders who
are upset at the unfairness of Wall Street and the stock market. The
members of r/WallStreetBets maintained a sense of unity and purpose
throughout the events (Muzio, 2021). Regardless of whether the organized
behavior of r/WallStreetBets users is a political movement, hopeful acts
of ``sticking it to hedge funds'' or something else, they have
demonstrated their power to move financial markets (Muzio, 2021).
However, retail investors as a whole both bought and ``shorted''--or bet
that GameStop stock was going to go down--indicating that ``the Gamestop
frenzy was not a pure digital protest against Wall Street but
speculative trading by a group of retail investors, in line with their
prior high-risk trading behavior'' (Hasso et al., 2021).

\hypertarget{rwallstreetbets}{%
\subsection{r/WallStreetBets}\label{rwallstreetbets}}

Wall Street Bets is a subreddit where investors talk about stocks. The
group members are known for their risky ``YOLO'' (You Only Live Once)
trades (Muzio, 2021). There is a particular language and meme culture
that permeates the group and potentially helps to create their culture
of ``degeneracy'' (Boylston et al., 2021). For example, they celebrate
people with ``diamond hands'' as those who are prepared to hold their
stocks for a long time and decry ``paper hands,'' those who miss out on
profits by selling a stock too early (Wallstreetbets News, 2020). They
maintain they are not ``dumb,'' but they are ``retarded'' or
``autists.'' Although these words are derogatory, they use them as a way
to self-deprecate and build unity.

\hypertarget{past-research-on-information-media-and-financial-markets}{%
\section{Past Research on Information, Media, and Financial
Markets}\label{past-research-on-information-media-and-financial-markets}}

Scholars have been interested in the role of information on financial
markets. Business researchers largely operate under the efficient market
hypothesis where markets reflect accurate valuations based on available
information. As such, they are concerned about whether information being
spread about companies is timely or ``stale'' and the effects of media
attention on stocks, including newspaper, television, digital news, and
social media.

\hypertarget{newspapers}{%
\subsection{Newspapers}\label{newspapers}}

Even though the role of newspapers plays an increasingly receding role
in the news ecosystem, media attention given to individual firms via
newspapers impacts the pricing of the stock. The causal link between
newspapers impacting trading prices and volumes was established by
leveraging newspaper strikes and observing the subsequent impact on
financial markets (Peress, 2014). This effect persists for many days
after publication, and the impact is prolonged during a recession
compared to an expansion (Antweiler \& Frank, 2006). However, not all
coverage is good coverage. Stocks with little or no attention in
newspapers earn higher returns than high-media-attention stocks (Fang \&
Peress, 2009).

There are two primary ways to define news: new versus stale and
quantitative versus qualitative. Stock market investors respond
differently to new versus stale news. New news is when news stories
provide original information about a firm. Stale news is when there is
additional content, but it does not provide new information.
Operationally in studies, new versus stale is defined by the textual
similarity of sequential newspaper stories. The prices of stocks respond
less to stale news, but there is a reversal in the movement of a stock's
price after stale news, suggesting that initial movements are
overreactions (Tetlock, 2011). Further, quantitative information is
easier for news readers to process compared to qualitative information
(Engelberg, 2008).

In addition, it is not just substantive news about a stock that impacts
performance. Stock recommendations in the \emph{Wall Street Journal's}
``Dartboard'' column, one of the original mass-consumer stock
recommendation media, predicted abnormal positive returns of 4\% and
double the trading volume over the two days after the column recommended
a stock (Barber \& Loeffler, 1993). These market abnormalities are
largely driven by naive investors and are ultimately noise in the
financial markets. Price movements from the column were reversed within
15 days, and investors following the ``expert'' recommendations of the
column lost 3.8\% on a risk-adjusted basis in the six months following
the recommendation (Liang, 1999). These findings show that media content
has a clear relationship with asset prices and should not be seen as a
``sideshow.'' They are also consistent with the theoretical models of
retail investors acting as noise in the financial ecosystem. And the
findings are inconsistent with the theory that media can serve as a
proxy for new financial information that informs the efficient market on
fundamental asset valuations (Tetlock, 2007).

While business scholars strictly look for how media content can drive
the price of stocks higher or lower, media psychologies also study the
psychological effects of content. For example, media psychologists study
the visceral effects of encountering violent news stories or
pornography. A similar approach can be taken to studying financial news
and positive versus negative stories of firms.

\hypertarget{television}{%
\subsection{Television}\label{television}}

Similar to newspapers, the financial markets react to media content on
television, often in real-time. The live nature of television has
shifted academic focus from new versus stale news and instead looks for
near-instantaneous market reactions. When the TV program ``The Morning
Call'' on \emph{CNBC} reports analyst views on individual stocks, the
market reacts within seconds, the price is fully incorporated within one
minute, and trading intensity in the stock doubles within that first
minute (Busse \& Clifton Green, 2002). ``Mad Money with Jim Cramer'' is
a popular \emph{CNBC} television show where the titular host, Jim
Cramer, makes a bevy of buy and sell recommendations on stocks. There
are significant price movements for stocks that Cramer recommends to buy
and sell (Bolster et al., 2012; Engelberg et al., 2012; Karniouchina et
al., 2009). The effects on the movement of stocks given a ``buy'' rating
from Cramer are quickly reversed, but the effect persists longer for
``sell'' recommendations (Bolster et al., 2012). The viewing ratings of
the show even predict the strength of the price movements (Engelberg et
al., 2012).

Business scholars and media psychologists take fundamentally different
approaches to studying television. Business scholars examine discrete
pieces of media in a vacuum--there is little consideration for broader
media ecology or content that is consumed before or after the specific
media content being studied. In the case of television, business
scholars will study a specific segment on television without concern for
the content before or after the specific segment. In contrast, media
psychologists study the increasing fragmentation of the media landscape.
The information just before a television segment can prime the media
consumer. Or, a television watcher can take out their cell phone and
look at related content while a media segment is ongoing. Business
scholars do not consider any of these media consumption realities.
Perhaps, controlling for these other factors can reveal an even stronger
effect of media on the prices of stocks.

\hypertarget{social-media}{%
\subsection{Social Media}\label{social-media}}

As evidenced by the recent GameStop and r/WallStreetBets saga, social
media is playing an increasing role in the way that individual investors
collect, share, and process information that results in investment
decisions. The literature on social media and the stock market falls
into three broad categories: social media as a predictor of stock
movements, crowd-sourced financial recommendations, and behavioral
contagion/ herding.

First, social media has been used to try to predict fluctuations in
individual stocks. The number of posts on TheRagingBull.com, one of the
original stock message boards, in 2001 coincided with days with
abnormally high trading volume, but they were not predictive of market
pricing or volume after controlling for industry-adjusted returns and
normal trading volume (Tumarkin \& Whitelaw, 2001). This result suggests
that some other exogenous events are confounders that drive both social
media engagement and stock market volume.

However, other new media, including Yahoo! stock message boards (Das \&
Chen, 2007) and Google Search Volume (Gao, 2011) are all predictive of
stock movements. Even disagreements on StockTwits, a Twitter-like
stock-based social media, predict price changes by affecting trading
done by uninformed investors and facilitates trading by informed traders
who are taking actions aimed at changing corporate policies (Cookson et
al., 2021).

Individual tweets just before a firm's quarterly earnings announcement
is a predictor of the earnings report and subsequent price action
(Mohanram, 2018). News diffusion on Twitter leads to lower bid-ask
spread and price pressure during a news day, but the effect is reversed
the next day (Ye, 2016). Twitter, like traditional media, spreads stale
news, albeit at a higher speed than traditional media (Ye, 2016).

Next, social media have been studied as a way to crowdsource financial
advice. For example, Seeking Alpha is a website where non-professional
investors can write articles analyzing a stock. Seeking Alpha articles
are predictors of stock market movements (Campbell et al., 2019; Chen et
al., 2014; Farrell et al., 2018). Estimize is an open platform that
solicits forecasts from contributors. These crowdsourced forecasts are
incrementally useful in forecasting earnings, and the higher the volume
of forecasts, the more accurate the predictions (Jame et al., 2016).
SumZero.com is a private social networking site that facilitates
interaction and information exchange among professional investors.
Recommendations offered on this website generate significant returns in
the financial markets (Crawford et al., 2018).

Finally, social media can create ``herding'' by which there is
behavioral contagion that flows through networks of investors via their
connections on social media. Social network links increase disposition
effects (Heimer, 2016), which is the tendency to sell ``winning'' assets
and hold onto ``losing'' assets (Shefrin \& Statman, 1985). Herding is
particularly prominent on Robinhood (Barber et al., 2021). And the
GameStop saga provides evidence for the idea of behavioral contagion
among r/WallStreetBets users (Semenova \& Winkler, 2021). On StockTwits
there is even evidence of echo chambers where users selectively expose
themselves to other users who share positive or negative sentiments on
specific stocks (Cookson et al., 2020). The herding behavior found in
online investor communities also happens in offline networks (Duflo \&
Saez, 2002, 2003; Ivković \& Weisbenner, 2007; Musciotto et al., 2018;
Shiller \& Pound, 1989). Potentially, this similarity between online and
offline social behavioral contagion can add to the digitization and
social network literature.

Herding behavior is very similar to how media psychologists have found
social norms to impact behavioral change. For example, making people
believe that all of their neighbors are recycling, an individual is more
likely to also recycle. Similarly, if an investor in an online community
perceives that everyone is purchasing GameStop stock, they are likely to
also participate in that newly-perceived social norm.

\hypertarget{conclusion-future-media-psychology-research}{%
\section{Conclusion \& Future Media Psychology
Research}\label{conclusion-future-media-psychology-research}}

Financial market scholars have intensely studied the impact that media
and information have on driving the prices of financial securities
higher and lower. However, approaching the topic from a media psychology
angle provides new questions and avenues of study. A psychological
approach places individual investors as the unit of analysis, not the
movements of financial markets.

\hypertarget{temporal-dynamics-of-stock-information}{%
\subsection{Temporal dynamics of stock
information}\label{temporal-dynamics-of-stock-information}}

It is unclear at what time intervals individual investors make trading
decisions. How long do retail investors take before purchasing a stock?
Do different types of information from sources of varying credibility
impact the scale of time in making a decision? Taking a Lemkeian
approach (2000) to understanding the temporal dimension of investment
decisions could help to answer these questions.

\hypertarget{gamification-and-design}{%
\subsection{Gamification and design}\label{gamification-and-design}}

Another area of future research is how the design of media and trading
platforms impacts financial decisions. There has been a lot of
speculation on how Robinhood's gamified interface spurs more and riskier
trades. How does the design of financial platforms and financial media
alter trading behaviors?

\hypertarget{addiction-escapism-and-video-games}{%
\subsection{Addiction, escapism, and video
games}\label{addiction-escapism-and-video-games}}

How is addiction to trading stocks similar or different than addiction
to other media, like video games where there is a constant feedback loop
of ``winning'' and ``losing'' (Bavelier et al., 2011)? Do people trade
stocks and consume financial information as a means of ``escaping'' and
taking on a persona as a big-time trader, just like video games are used
for escapism (Granic et al., 2014; Reeves \& Read, 2009)?

\hypertarget{social-contagion}{%
\subsection{Social contagion}\label{social-contagion}}

How does trading behavior spread across networks of people? We know that
behavior does spread, but media psychologists can help answer
\emph{how}, similar to studies on emotional contagion in social networks
(Kramer et al., 2014). Past research has focused on aggregated groups of
people, for example working in the same office or in the same geographic
area (Duflo \& Saez, 2002, 2003; Ivković \& Weisbenner, 2007; Musciotto
et al., 2018; Shiller \& Pound, 1989). Semenova and Winkler (2021)
provide a good start on modeling individual trading decisions within
digital networks, but how can media psychologists add their knowledge of
interpersonal dynamics to mathematical modeling of behavioral contagion?

\hypertarget{media-fragmentation-and-information-gathering}{%
\subsection{Media fragmentation and information
gathering}\label{media-fragmentation-and-information-gathering}}

Finally, digital technologies have fragmented the media experiences of
users. Instead of sitting down at one's computer and intensely trading
stocks, many users now pull up a trading app on their phone and casually
trade. Are most people using the Robinhood app for hours at a time?
Unlikely, instead, it is more probable that there is media fragmentation
that causes multitasking or task switching which creates short bursts of
attention given to a trading app intertwined with other media. Can
interacting with other types of media impact how one trades? Does
trading behavior impact the other media that a trader consumes? How do
traders respond to massive financial gains and losses, either of the
entire market or their individual portfolio, similar to the study of
individuals' responses to television imagery (Newhagen \& Reeves, 1992)?
How do people even go about getting information to make a trading
decision? How is trust built between someone like Jim Cramer or a random
Reddit user and investors? All of these questions remain unanswered.

\hypertarget{conclusion}{%
\subsection{Conclusion}\label{conclusion}}

Scholars of financial markets have intensely studied the psychology of
investing and personal finance. They have also studied relationships
between media and financial markets. But all of their research questions
revolve around studying the market as the outcome. Media psychologists
can add to this area of research by studying individual behavior as the
dependent variable.

\hypertarget{references}{%
\section*{References}\label{references}}
\addcontentsline{toc}{section}{References}

\hypertarget{refs}{}
\leavevmode\hypertarget{ref-antweiler2006}{}%
Antweiler, W., \& Frank, M. Z. (2006). \emph{Do us stock markets
typically overreact to corporate news stories}.

\leavevmode\hypertarget{ref-barber2021}{}%
Barber, B. M., Huang, X., Odean, T., \& Schwarz, C. (2021). Attention
induced trading and returns: Evidence from robinhood users.
\emph{Available at SSRN: Https://Ssrn.com/Abstract=3715077 or
Http://Dx.doi.org/10.2139/Ssrn.3715077}.

\leavevmode\hypertarget{ref-barber1993}{}%
Barber, B. M., \& Loeffler, D. (1993). The "dartboard" column:
Second-hand information and price pressure. \emph{The Journal of
Financial and Quantitative Analysis}, \emph{28}(2), 273--284.
\url{http://www.jstor.org/stable/2331290}

\leavevmode\hypertarget{ref-bavelier2011}{}%
Bavelier, D., Green, C. S., Han, D. H., Renshaw, P. F., \& Merzenich, M.
M. (2011). Brains on video games. \emph{Nature Reviews Neuroscience},
\emph{12}, 763--768.
\url{https://doi.org/https://doi.org/10.1038/nrn3135}

\leavevmode\hypertarget{ref-bolster2012}{}%
Bolster, P., Trahan, E., \& Venkateswaran, A. (2012). How mad is mad
money? Jim cramer as a stock picker and portfolio manager. \emph{The
Journal of Investing}, \emph{21}(2), 27--39.
\url{https://doi.org/10.3905/joi.2012.21.2.027}

\leavevmode\hypertarget{ref-boylston2021}{}%
Boylston, C., Palacios, B., Tassev, P., \& Bruckman, A. (2021).
\emph{WallStreetBets: Positions or ban}.
\url{http://arxiv.org/abs/2101.12110}

\leavevmode\hypertarget{ref-busse2002}{}%
Busse, J. A., \& Clifton Green, T. (2002). Market efficiency in real
time. \emph{Journal of Financial Economics}, \emph{65}(3), 415--437.
\url{https://doi.org/https://doi.org/10.1016/S0304-405X(02)00148-4}

\leavevmode\hypertarget{ref-campbell2019}{}%
Campbell, J. L., DeAngelis, M. D., \& Moon, J. R. (2019). Skin in the
game: personal stock holdings and investors' response to stock analysis
on social media. \emph{Review of Accounting Studies}, \emph{24}(3),
731--779. \url{https://doi.org/10.1007/s11142-019-09498-}

\leavevmode\hypertarget{ref-chen2014}{}%
Chen, H., De, P., Hu, Y. (., \& Hwang, B.-H. (2014). Wisdom of Crowds:
The Value of Stock Opinions Transmitted Through Social Media. \emph{The
Review of Financial Studies}, \emph{27}(5), 1367--1403.
\url{https://doi.org/10.1093/rfs/hhu001}

\leavevmode\hypertarget{ref-cookson2020}{}%
Cookson, J. A., Engelberg, J. E., \& Mullins, W. (2020). Echo chambers.
\emph{SocArXiv1}. \url{https://doi.org/doi:10.31235/osf.io/n2q9h}

\leavevmode\hypertarget{ref-cookson2021}{}%
Cookson, J. A., Fos, V., \& Niessner, M. (2021). Does disagreement
facilitate informed trading? Evidence from activist investors.
\emph{Available at SSRN: Https://Ssrn.com/Abstract=3765092 or
Http://Dx.doi.org/10.2139/Ssrn.3765092}.

\leavevmode\hypertarget{ref-crawford2018}{}%
Crawford, S., Gray, W., Johnson, B. R., \& Price, R. A. (2018). What
Motivates Buy-Side Analysts to Share Recommendations Online?
\emph{Management Science}, \emph{64}(6), 2574--2589.
\url{https://doi.org/10.287/mnsc.2017.2749}

\leavevmode\hypertarget{ref-das2007}{}%
Das, S. R., \& Chen, M. Y. (2007). Yahoo! for Amazon: Sentiment
Extraction from Small Talk on the Web. \emph{Management Science},
\emph{53}(9), 1375--1388. \url{https://doi.org/10.1287/mnsc.1070.0704}

\leavevmode\hypertarget{ref-duflo2002}{}%
Duflo, E., \& Saez, E. (2002). Participation and investment decisions in
a retirement plan: the influence of colleagues' choices. \emph{Journal
of Public Economics}, \emph{85}(1), 121--148.
\url{https://ideas.repec.org/a/eee/pubeco/v85y2002i1p121-148.html}

\leavevmode\hypertarget{ref-duflo2003}{}%
Duflo, E., \& Saez, E. (2003). The Role of Information and Social
Interactions in Retirement Plan Decisions: Evidence from a Randomized
Experiment*. \emph{The Quarterly Journal of Economics}, \emph{118}(3),
815--842. \url{https://doi.org/10.1162/00335530360698432}

\leavevmode\hypertarget{ref-engelberg2008}{}%
Engelberg, J. (2008). Costly information processing: Evidence from
earnings announcements. \emph{AFA 2009 San Francisco Meetings Paper}.
\url{https://doi.org/http://dx.doi.org/10.2139/ssrn.1107998\%20}

\leavevmode\hypertarget{ref-engelberg2012}{}%
Engelberg, J., Sasseville, C., \& Williams, J. (2012). Market madness?
The case of "mad money". \emph{Management Science}, \emph{58}(2),
351--364. \url{http://www.jstor.org/stable/41406393}

\leavevmode\hypertarget{ref-fang2009}{}%
Fang, L., \& Peress, J. (2009). Media coverage and the cross-section of
stock returns. \emph{The Journal of Finance}, \emph{64}(5), 2023--2052.
\url{https://doi.org/https://doi.org/10.1111/j.1540-6261.2009.01493.x}

\leavevmode\hypertarget{ref-farrell2018}{}%
Farrell, M., Green, T. C., Jame, R., \& Markov, S. (2018). The
democratization of investment research and the informativeness of retail
investor trading. \emph{Available at SSRN:
Https://Ssrn.com/Abstract=3222841 or
Http://Dx.doi.org/10.2139/Ssrn.3222841}.

\leavevmode\hypertarget{ref-da2011}{}%
Gao, Z. D. J. E. P. (2011). In search of attention. \emph{The Journal of
Finance}, \emph{66}(5), 1461--1499.
\url{https://doi.org/https://doi.org/10.1111/j.1540-6261.2011.01679.x}

\leavevmode\hypertarget{ref-granic2014}{}%
Granic, I., Lobel, A., \& Engels, R. (2014). The benefits of playing
video games. \emph{The American Psychologist}, \emph{69 1}, 66--78.

\leavevmode\hypertarget{ref-hasso2021}{}%
Hasso, T., Müller, D., Pelster, M., \& Warkulat, S. (2021). Who
participated in the gamestop frenzy? \emph{TAF Working Paper No. 58}.

\leavevmode\hypertarget{ref-heimer2016}{}%
Heimer, R. Z. (2016). Peer Pressure: Social Interaction and the
Disposition Effect. \emph{The Review of Financial Studies},
\emph{29}(11), 3177--3209. \url{https://doi.org/10.1093/rfs/hhw063}

\leavevmode\hypertarget{ref-ivkovic2007}{}%
Ivković, Z., \& Weisbenner, S. (2007). Information Diffusion Effects in
Individual Investors' Common Stock Purchases: Covet Thy Neighbors'
Investment Choices. \emph{The Review of Financial Studies},
\emph{20}(4), 1327--1357. \url{https://doi.org/10.1093/revfin/hhm009}

\leavevmode\hypertarget{ref-jame2016}{}%
Jame, R., Johnston, R., Markov, S., \& Wolfe, M. C. (2016). The value of
crowdsourced earnings forecasts. \emph{Journal of Accounting Research},
\emph{54}(4), 1077--1110.
\url{https://doi.org/https://doi.org/10.1111/1475-679X.12121}

\leavevmode\hypertarget{ref-kannadhasan2015}{}%
Kannadhasan, M. (2015). Retail investors' financial risk tolerance and
their risk-taking behaviour: The role of demographics as differentiating
and classifying factors. \emph{IIMB Management Review}, \emph{27}(3),
175--184.
\url{https://doi.org/https://doi.org/10.1016/j.iimb.2015.06.004}

\leavevmode\hypertarget{ref-karniouchina2009}{}%
Karniouchina, E. V., Moore, W. L., \& Cooney, K. J. (2009). Impact of
"mad money" stock recommendations: Merging financial and marketing
perspectives. \emph{Journal of Marketing}, \emph{73}(6), 244--266.
\url{http://www.jstor.org/stable/20619072}

\leavevmode\hypertarget{ref-kramer2014}{}%
Kramer, A. D. I., Guillory, J. E., \& Hancock, J. T. (2014).
Experimental evidence of massive-scale emotional contagion through
social networks. \emph{Proceedings of the National Academy of Sciences},
\emph{111}(24), 8788--8790.
\url{https://doi.org/10.1073/pnas.1320040111}

\leavevmode\hypertarget{ref-lemke}{}%
Lemke, J. L. (2000). Across the scales of time: Artifacts, activities,
and meanings in ecosocial systems. \emph{Mind, Culture, and Activity},
\emph{7}(4), 273--290.
\url{https://doi.org/10.1207/S15327884MCA0704/_03}

\leavevmode\hypertarget{ref-liang1999}{}%
Liang, B. (1999). Price Pressure: Evidence from the
\&quot;Dartboard\&quot; Column. \emph{The Journal of Business},
\emph{72}(1), 119--134. \url{https://doi.org/10.1086/209604}

\leavevmode\hypertarget{ref-lopatto2021}{}%
Lopatto, E. (2021). How r/wallstreetbets gamed the stock of gamestop.
\emph{The Verge}.
\url{https://www.theverge.com/22251427/reddit-gamestop-stock-short-wallstreetbets-robinhood-wall-street}

\leavevmode\hypertarget{ref-lyocsa2021}{}%
Lyócsa, Š., Baumöhl, E., \& Vŷrost, T. (2021). YOLO trading: Riding the
herd during the gamestop episode. \emph{ZBW - Leibniz Information Centre
for Economics, Working Paper}. \url{http://hdl.handle.net/10419/230679}

\leavevmode\hypertarget{ref-macey2021}{}%
Macey, J. R. (2021). Securities regulation as class warfare.
\emph{Columbia Business Law Review, Forthcoming}.
\url{https://doi.org/http://dx.doi.org/10.2139/ssrn.3789706}

\leavevmode\hypertarget{ref-bartov2018}{}%
Mohanram, E. B. L. F. P. S. (2018). Can twitter help predict firm-level
earnings and stock returns? \emph{The Accounting Review}, \emph{93}(3),
25--57. \url{https://doi.org/https://doi.org/10.2308/accr-51865}

\leavevmode\hypertarget{ref-musciotto2018}{}%
Musciotto, F., Marotta, L., Piilo, J., \& Mantegna, R. N. (2018).
Long-term ecology of investors in a financial market. \emph{Palgrave
Communication}, \emph{4}(1), 92.

\leavevmode\hypertarget{ref-dimuzio2021}{}%
Muzio, T. D. (2021). GameStop capitalism. Wall street vs. The reddit
rally (part 1). \emph{University of Wollongong Working Paper}, 1--13.
\url{http://bnarchives.yorku.ca/673/}

\leavevmode\hypertarget{ref-newhagen1992}{}%
Newhagen, J. E., \& Reeves, B. (1992). The evening's bad news: Effects
of compelling negative television news images on memory. \emph{Journal
of Communication}, \emph{42}(2), 25--41.
\url{https://doi.org/https://doi.org/10.1111/j.1460-2466.1992.tb00776.x}

\leavevmode\hypertarget{ref-eaton2021}{}%
Pagano, G. W., Green, C. T., Roseman, B., \& Wu, Y. (2021).
Zero-commission individual investors, high frequency traders, and stock
market quality. \emph{Available at SSRN:
Https://Ssrn.com/Abstract=3776874 or
Http://Dx.doi.org/10.2139/Ssrn.3776874}.
\url{https://doi.org/http://dx.doi.org/10.2139/ssrn.3776874}

\leavevmode\hypertarget{ref-pagano2020}{}%
Pagano, M. S., Sedunov, J., \& Velthuis, R. (2020). How did retail
investors respond to the covid-19 pandemic? The effect of robinhood
brokerage customers on market quality. \emph{Finance Research Letters,
Forthcoming}.
\url{https://doi.org/http://dx.doi.org/10.2139/ssrn.3703815\%20}

\leavevmode\hypertarget{ref-peress2014}{}%
Peress, J. (2014). The media and the diffusion of information in
financial markets: Evidence from newspaper strikes. \emph{The Journal of
Finance}, \emph{69}(5), 2007--2043.
\url{https://doi.org/https://doi.org/10.1111/jofi.12179}

\leavevmode\hypertarget{ref-reeves2009}{}%
Reeves, B., \& Read, J. L. (2009). \emph{Total engagement: Using games
and virtual worlds to change the way people work and businesses
compete}.

\leavevmode\hypertarget{ref-regnier2021}{}%
Regnier, P. (2021). Stonks are bonkers, and other lessons from the
reddit rebellion. \emph{Bloomberg.com}.
\url{https://www.bloomberg.com/news/features/2021-02-04/gamestop-gme-how-wallstreetbets-and-robinhood-created-bonkers-stock-market}

\leavevmode\hypertarget{ref-semenova2021}{}%
Semenova, V., \& Winkler, J. (2021). Reddit's self-organized bull runs.
\emph{Available at Https://Mpra.ub.uni-Muenchen.de/105630/}.

\leavevmode\hypertarget{ref-shefrin1985}{}%
Shefrin, H., \& Statman, M. (1985). The disposition to sell winners too
early and ride losers too long: Theory and evidence. \emph{The Journal
of Finance}, \emph{40}(3), 777--790.
\url{https://doi.org/https://doi.org/10.1111/j.1540-6261.1985.tb05002.x}

\leavevmode\hypertarget{ref-shiller1989}{}%
Shiller, 0. J., \& Pound, J. (1989). Survey evidence on diffusion of
interest and information among investors. \emph{Journal of Economic
Behavior \& Organization}, \emph{12}(1), 47--66.
\url{https://ideas.repec.org/a/eee/jeborg/v12y1989i1p47-66.html}

\leavevmode\hypertarget{ref-talwar2021}{}%
Talwar, M., Talwar, S., Kaur, P., Tripathy, N., \& Dhir, A. (2021). Has
financial attitude impacted the trading activity of retail investors
during the covid-19 pandemic? \emph{Journal of Retailing and Consumer
Services}, \emph{58}, 102341.
\url{https://doi.org/https://doi.org/10.1016/j.jretconser.2020.102341}

\leavevmode\hypertarget{ref-tetlock2007}{}%
Tetlock, P. C. (2007). Giving content to investor sentiment: The role of
media in the stock market. \emph{The Journal of Finance}, \emph{62}(3),
1139--1168.
\url{https://doi.org/https://doi.org/10.1111/j.1540-6261.2007.01232.x}

\leavevmode\hypertarget{ref-tetlock2011}{}%
Tetlock, P. C. (2011). All the News That's Fit to Reprint: Do Investors
React to Stale Information? \emph{The Review of Financial Studies},
\emph{24}(5), 1481--1512. \url{https://doi.org/10.1093/rfs/hhq141}

\leavevmode\hypertarget{ref-tumarkin2001}{}%
Tumarkin, R., \& Whitelaw, R. F. (2001). News or noise? Internet
postings and stock prices. \emph{Financial Analysts Journal},
\emph{57}(3), 41--51. \url{https://doi.org/10.2469/faj.v57.n3.2449}

\leavevmode\hypertarget{ref-wsbshop}{}%
Wallstreetbets News. (2020). \emph{Dissecting the unique lingo and
terminolgy used in the subreddit r/wallstreetbets}.
\url{\%22https://www.wallstreetbets.shop/blogs/news/dissecting-the-unique-lingo-and-terminology-used-in-the-subreddit-r-wallstreetbets\%22}

\leavevmode\hypertarget{ref-welch2020}{}%
Welch, I. (2020). The wisdom of the robinhood crowd. \emph{NBER Working
Paper Series}. \url{http://www.nber.org/papers/w27866}

\leavevmode\hypertarget{ref-chawla2016}{}%
Ye, N. C. Z. D. J. X. M. (2016). Information diffusion on social media:
Does it affect trading, return, and liquidity. \emph{Available at SSRN:
Https://Ssrn.com/Abstract=2935138 or
Http://Dx.doi.org/10.2139/Ssrn.2935138}.





\newpage
\singlespacing 
\end{document}
