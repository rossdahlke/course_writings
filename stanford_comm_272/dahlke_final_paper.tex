\documentclass[12pt,]{article}
\usepackage[left=1in,top=1in,right=1in,bottom=1in]{geometry}
\newcommand*{\authorfont}{\fontfamily{phv}\selectfont}
\usepackage[]{mathpazo}


  \usepackage[T1]{fontenc}
  \usepackage[utf8]{inputenc}




\usepackage{abstract}
\renewcommand{\abstractname}{}    % clear the title
\renewcommand{\absnamepos}{empty} % originally center

\renewenvironment{abstract}
 {{%
    \setlength{\leftmargin}{0mm}
    \setlength{\rightmargin}{\leftmargin}%
  }%
  \relax}
 {\endlist}

\makeatletter
\def\@maketitle{%
  \newpage
%  \null
%  \vskip 2em%
%  \begin{center}%
  \let \footnote \thanks
    {\fontsize{18}{20}\selectfont\raggedright  \setlength{\parindent}{0pt} \@title \par}%
}
%\fi
\makeatother




\setcounter{secnumdepth}{0}

\usepackage{longtable,booktabs}



\title{Retail investors and media psychology  }



\author{\Large Ross Dahlke\vspace{0.05in} \newline\normalsize\emph{}  }


\date{}

\usepackage{titlesec}

\titleformat*{\section}{\normalsize\bfseries}
\titleformat*{\subsection}{\normalsize\itshape}
\titleformat*{\subsubsection}{\normalsize\itshape}
\titleformat*{\paragraph}{\normalsize\itshape}
\titleformat*{\subparagraph}{\normalsize\itshape}





\newtheorem{hypothesis}{Hypothesis}
\usepackage{setspace}


% set default figure placement to htbp
\makeatletter
\def\fps@figure{htbp}
\makeatother

\usepackage{graphicx}

% move the hyperref stuff down here, after header-includes, to allow for - \usepackage{hyperref}

\makeatletter
\@ifpackageloaded{hyperref}{}{%
\ifxetex
  \PassOptionsToPackage{hyphens}{url}\usepackage[setpagesize=false, % page size defined by xetex
              unicode=false, % unicode breaks when used with xetex
              xetex]{hyperref}
\else
  \PassOptionsToPackage{hyphens}{url}\usepackage[draft,unicode=true]{hyperref}
\fi
}

\@ifpackageloaded{color}{
    \PassOptionsToPackage{usenames,dvipsnames}{color}
}{%
    \usepackage[usenames,dvipsnames]{color}
}
\makeatother
\hypersetup{breaklinks=true,
            bookmarks=true,
            pdfauthor={Ross Dahlke ()},
             pdfkeywords = {},  
            pdftitle={Retail investors and media psychology},
            colorlinks=true,
            citecolor=blue,
            urlcolor=blue,
            linkcolor=magenta,
            pdfborder={0 0 0}}
\urlstyle{same}  % don't use monospace font for urls

% Add an option for endnotes. -----


% add tightlist ----------
\providecommand{\tightlist}{%
\setlength{\itemsep}{0pt}\setlength{\parskip}{0pt}}

% add some other packages ----------

% \usepackage{multicol}
% This should regulate where figures float
% See: https://tex.stackexchange.com/questions/2275/keeping-tables-figures-close-to-where-they-are-mentioned
\usepackage[section]{placeins}


\begin{document}
	
% \pagenumbering{arabic}% resets `page` counter to 1 
%
% \maketitle

{% \usefont{T1}{pnc}{m}{n}
\setlength{\parindent}{0pt}
\thispagestyle{plain}
{\fontsize{18}{20}\selectfont\raggedright 
\maketitle  % title \par  

}

{
   \vskip 13.5pt\relax \normalsize\fontsize{11}{12} 
\textbf{\authorfont Ross Dahlke} \hskip 15pt \emph{\small }   

}

}






\vskip -8.5pt


 % removetitleabstract

\noindent \doublespacing 

\begin{longtable}[]{@{}lll@{}}
\toprule
Method & koRpus & stringi\tabularnewline
\midrule
\endhead
Word count & 1643 & 1657\tabularnewline
Character count & 10971 & 10970\tabularnewline
Sentence count & 79 & Not available\tabularnewline
Reading time & 8.2 minutes & 8.3 minutes\tabularnewline
\bottomrule
\end{longtable}

\hypertarget{abstract}{%
\section{Abstract}\label{abstract}}

\hypertarget{introduction}{%
\section{Introduction}\label{introduction}}

Financial markets were recently upended by massive short squeezes of a
cadre of stocks including GameStop, AMC Entertainment Holdings,
BlackBerry, and Nokia. The recent events saw historic fluctuations in
the prices of these stocks. Individual investors made tens-of-millions
of dollars, often with leveraged positions. What is particularly unique
about these events is that they were triggered by retail investors.
These retail investors, bolstered by financial stimuli and an abundance
of free time due to the COVID-19 pandemic, organized collective behavior
actions on an internet forum, the Reddit subforum r/WallStreetBets.

Media commentators quickly offered various perspectives on the events.
Is this class warfare? A populist movement? Just some bored, lonely
young men trying to make money? There is a deep pool of academic
literature that explores the psychological motivations of retail
investors and personal finance more broadly. For example,
\_\_\_\_\_\_\_\_\_. In addition, there is a small, yet robust and
powerful set of research on the impacts of media on financial markets.
However, I am interest in the media psychology component of financial
decisions. I am interested in how media impact financial markets
in-so-much-as how the media effects, and is driven by, individual
investors.

More broadly, my research investigates how individuals enter and
participate in complex systems. In this case, the totality of the
financial markets are beyond comprehension. Investors, particularly
small, retail traders, are unlikely to ever fully understand the complex
system of global finance. Yet, they do. And as is increasingly common,
participants are coordinating activity in order to alter the systems
that they participate in. As it pertains to retail investors, I
ultimately want to know both \emph{why} they participate in the markets,
and more importantly, \emph{how} they participate in the financial
markets. What is the media ecology that individual investors are exposed
to or seek out? How do novice traders come to trust the financial
investment advice of various media outlets, communities, or individuals?
What do people do after they lose their life savings in GameStop stock?
How much time do retail investors devote to trading financial
securities?

\hypertarget{covid-19-and-retail-traders}{%
\subsection{COVID-19 and Retail
Traders}\label{covid-19-and-retail-traders}}

The COVID-19 global pandemic has spurred a rise in retail investors that
rivals the boom of retail investors during the internet bubble after the
rise of discount stock brokerages. At the center of the rise in retail
investors is Robinhood. Robinhood is a smart phone app and stock trading
platform that offers free trading of stocks and other securities. While
other trading platforms have since adopted free trading, Robinhood
created and maintained a significant following by being the first to
offer free trading, it's gamified user interface, and a target
demographic of young, new investors. With Robinhood easing barriers of
entry for new investors, financial stimuli from the government due to
the pandemic, and newly-found swathes of free-time, these retail
investors grew in size and influence during the pandemic.

Retail investors now routinely account for 20\% of stock market activity
(Pagano et al. 2021) These retail investors on Robinhood significantly
impact financial markets, particularly during COVID (Pagano, Sedunov,
and Velthuis 2020). Users of Robinhood increased the amount of money
they invested on the platform during the pandemic (Welch 2020).
Robinhood traders primarily engage in both momentum and contrarian
strategies where they invest in stocks that have already demonstrated
rising price momentum and ``buying the dip'' where they buy stocks that
have recently fallen in price, respectively (Pagano, Sedunov, and
Velthuis 2020). In other words, Robinhood investors did not panic during
broad market turmoil during COVID-19 and often used price drops as
buying opportunities. However, in aggregate they do not produce an
alpha, or ``beat the market'' by producing larger-than-average returns
(Pagano et al. 2021). Their trades ultimately mostly produce noise in
financial markets (Pagano et al. 2021). If all Robinhood investors do is
create ``noise'' in the markets, then the noisiest they've been is when
GameStop stock rose dramatically in early 2021.

\hypertarget{gamestop}{%
\subsection{GameStop}\label{gamestop}}

In January and February 2021, a handful of stocks experienced an extreme
``short-squeeze'' which effectively sky-rocketed the prices of these
stocks that was caused by decentralized retail investors acting in
concert (Lyócsa, Baumöhl, and Vŷrost 2021). The largest movement was
seen in GameStop which took most of the attention of the saga. Many
investors, both individual and institutional, gained and lost
significant sums of money. For a full recapping of the sequence of
events, I recommend \_\_\_\_\_\_\_\_\_.

The GameStop saga provides support for the principle that market forces
tend to make markets fair, ``where fairness is defined as investors
`getting what they pay for' rather than investors `beating the
market'\,'' due to the apparent misvaluation of GameStop and other
stocks by institutional investors (Macey 2021). On the other hand, the
events also cast doubt as to whether the SEC is achieving its stated
goal of maintaining fair, orderly, and efficient markets" (Macey 2021).

The retail investors who caused the short squeeze organized on a
subforum of the popular website of Reddit called ``Wall Street Bets,''
or r/wallstreetbets (Lyócsa, Baumöhl, and Vŷrost 2021). As a result, the
GameStop saga is seen as battle between Wall Street and small, Robinhood
traders who are upset at Wall Street. The members of r/wallstreetbets
maintained a sense of unity and purpose throughout the events (Muzio
2021). Regardless of whether the organized behavior of r/wallstreetbets
users is a political movement, hopeful acts of ``sticking it to
hedgefunds'' or something else, they have demonstrated their power to
move financial markets (Muzio 2021). However, retail investors as whole
both bought and ``shorted''--or bet that GameStop stock was going to go
down--indicating that ``the Gamestop frenzy was not a pure digital
protest against Wall Street but speculative trading by a group of retail
investors, in line with their prior high-risk trading behavior'' (Hasso
et al. 2021).

\hypertarget{rwallstreetbets}{%
\subsection{r/WallStreetBets}\label{rwallstreetbets}}

Wall Street Bets is a subreddit where investors talk about their stock
trades. The group is known for their risky ``YOLO'' (You Only Live Once)
trades (Muzio 2021). There is a particular language and meme culture
that permeates from the group and potentially helps to create their
culture of ``degeneracy'' (Boylston et al. 2021). For example, they
celebrate people with ``diamond hands'' as those who are prepared to
hold to their stocks for a long time and decry ``paper hands,'' those
who miss out on profits by selling too a stock too early (Wallstreetbets
News 2020). They maintain they are not ``dumb,'' but they are
``retarded'' or ``autists.'' Although these words are derogatory, they
use it as a way to self-deprecate and build unity.

\hypertarget{information-and-media}{%
\section{Information and Media}\label{information-and-media}}

Scholars have been particularly interested with the role of information
on financial markets. Business researchers largely operate under the
efficient market hypothesis where markets reflect accurate valuations
based on available information. As such, they are primarily interested
in whether information being spread about companies is timely or
``stale'' and the effects of media attention on stocks, media including
television, newspaper, digital news, and social media on financial
markets. Overall, media attention-driven behavior favors individuals
buying stocks opposed to selling stocks (Barber and Odean 2007).

\hypertarget{newspapers}{%
\subsection{Newspapers}\label{newspapers}}

Even though one may think of the stock market reacting to news
instantaneously, media attention given to individual stocks via
newspapers impact the pricing of the stock. The causal link between
newspapers impacting trading prices and volumes was established by
leveraging newspaper strikes and observing the subsequent impact on
financial markets (Peress 2014). This effect persists for many days
after public, and the impact is prolonged during a recession compared to
an expansion (Antweiler and Frank 2006). However, not all coverage is
good coverage. Stocks with little or no attention in newspapers earn
higher returns than high-media-attention stocks (Fang and Peress 2009).

Stock market investors respond differently to new versus stale news--as
defined by textual similarity of sequential newspaper stories. The
prices of stocks respond less to stale news, but there is a reversal in
the movement of a stock's price after stale news, suggesting that
initial movements are overreactions (Tetlock 2011). Further,
quantitative information is easier for news readers to process compared
to qualitative information (Engelberg 2008).

In addition, it is not just substantive news about a stock that impacts
performance, stock recommendations in the \emph{Wall Street Journal's}
``Dartboard'' column predicted abnormal positive returns of 4\% and
double the trading volume over the two days after the column recommended
a stock (Barber and Loeffler 1993). These market abnormalities are
largely driven by naive investors and are ultimately noise in the
financial markets. Price movements from the column were reversed within
15 days, and investors following the ``expert'' recommendations of the
column lost 3.8\% on a risk-adjusted basis in the six months following
the recommendation (Liang 1999). These findings show that media content
has a clear relationship with asset prices and should not be seen as a
``sideshow,'' are consistent with the theoretical models of retail
investors acting as noise in the financial ecosystem, and is
inconsistent with the theory that media can serve as a proxy for new
financial information that informs the efficient market on fundamental
asset valuations (Tetlock 2007).

\hypertarget{television}{%
\subsection{Television}\label{television}}

Similar to newspapers, the financial markets react to media content on
television, often in real-time. When the TV program ``The Morning Call''
on \emph{CNBC} reports analyst views on individuals stocks, the market
reacts within seconds, the price is fully incorporated within one
minute, and trading intensity in the stock doubles within that first
minute (Busse and Clifton Green 2002). ``Mad Money with Jim Cramer'' is
a popular \emph{CNBC} television show where the host, Jim Cramer, makes
a bevy of buy and sell recommendations on stocks. There are significant
price movements for stocks that Cramer recommends to buy and sell
(Bolster, Trahan, and Venkateswaran 2012; Engelberg, Sasseville, and
Williams 2012; Karniouchina, Moore, and Cooney 2009). The effects on the
movement of stocks given a ``buy'' rating from Cramer are quickly
reversed, but the effect persists longer for ``sell'' recommendations
(Bolster, Trahan, and Venkateswaran 2012). The viewing ratings of the
show even predicts the strength of the price movements (Engelberg,
Sasseville, and Williams 2012). Further, tradition advertising
variables, such as message length, recency-primacy effects, information
clutter, and source credibility all influence the size of the market
reaction to a ``buy'' recommendation from Cramer (Karniouchina, Moore,
and Cooney 2009), which potentially suggests that content consumers
process media from financial pundits similarly to other types of
advertising.

\hypertarget{social-media}{%
\subsection{Social Media}\label{social-media}}

\hypertarget{references}{%
\section*{References}\label{references}}
\addcontentsline{toc}{section}{References}

\hypertarget{refs}{}
\leavevmode\hypertarget{ref-antweiler2006}{}%
Antweiler, Werner, and Murray Z. Frank. 2006. ``Do Us Stock Markets
Typically Overreact to Corporate News Stories.'' In.

\leavevmode\hypertarget{ref-barber1993}{}%
Barber, Brad M., and Douglas Loeffler. 1993. ``The "Dartboard" Column:
Second-Hand Information and Price Pressure.'' \emph{The Journal of
Financial and Quantitative Analysis} 28 (2): 273--84.
\url{http://www.jstor.org/stable/2331290}.

\leavevmode\hypertarget{ref-barber2007}{}%
Barber, Brad M., and Terrance Odean. 2007. ``All That Glitters: The
Effect of Attention and News on the Buying Behavior of Individual and
Institutional Investors.'' \emph{The Review of Financial Studies} 21
(2): 785--818. \url{https://doi.org/10.1093/rfs/hhm079}.

\leavevmode\hypertarget{ref-bolster2012}{}%
Bolster, Paul, Emery Trahan, and Anand Venkateswaran. 2012. ``How Mad Is
Mad Money? Jim Cramer as a Stock Picker and Portfolio Manager.''
\emph{The Journal of Investing} 21 (2): 27--39.
\url{https://doi.org/10.3905/joi.2012.21.2.027}.

\leavevmode\hypertarget{ref-boylston2021}{}%
Boylston, Christian, Beatriz Palacios, Plamen Tassev, and Amy Bruckman.
2021. ``WallStreetBets: Positions or Ban.''
\url{http://arxiv.org/abs/2101.12110}.

\leavevmode\hypertarget{ref-busse2002}{}%
Busse, Jeffrey A., and T. Clifton Green. 2002. ``Market Efficiency in
Real Time.'' \emph{Journal of Financial Economics} 65 (3): 415--37.
\url{https://doi.org/https://doi.org/10.1016/S0304-405X(02)00148-4}.

\leavevmode\hypertarget{ref-engelberg2008}{}%
Engelberg, Joseph. 2008. ``Costly Information Processing: Evidence from
Earnings Announcements.'' \emph{AFA 2009 San Francisco Meetings Paper},
January.
\href{https://doi.org/http://dx.doi.org/10.2139/ssrn.1107998\%20}{https://doi.org/http://dx.doi.org/10.2139/ssrn.1107998}.

\leavevmode\hypertarget{ref-engelberg2012}{}%
Engelberg, Joseph, Caroline Sasseville, and Jared Williams. 2012.
``Market Madness? The Case of "Mad Money".'' \emph{Management Science}
58 (2): 351--64. \url{http://www.jstor.org/stable/41406393}.

\leavevmode\hypertarget{ref-fang2009}{}%
Fang, Lily, and Joel Peress. 2009. ``Media Coverage and the
Cross-Section of Stock Returns.'' \emph{The Journal of Finance} 64 (5):
2023--52.
\url{https://doi.org/https://doi.org/10.1111/j.1540-6261.2009.01493.x}.

\leavevmode\hypertarget{ref-hasso2021}{}%
Hasso, Tim, Daniel Müller, Matthias Pelster, and Sonja Warkulat. 2021.
``Who Participated in the Gamestop Frenzy?'' \emph{TAF Working Paper No.
58}, February.

\leavevmode\hypertarget{ref-karniouchina2009}{}%
Karniouchina, Ekaterina V., William L. Moore, and Kevin J. Cooney. 2009.
``Impact of "Mad Money" Stock Recommendations: Merging Financial and
Marketing Perspectives.'' \emph{Journal of Marketing} 73 (6): 244--66.
\url{http://www.jstor.org/stable/20619072}.

\leavevmode\hypertarget{ref-liang1999}{}%
Liang, Bing. 1999. ``Price Pressure: Evidence from the
\&quot;Dartboard\&quot; Column.'' \emph{The Journal of Business} 72 (1):
119--34. \url{https://doi.org/10.1086/209604}.

\leavevmode\hypertarget{ref-lyocsa2021}{}%
Lyócsa, Štefan, Eduard Baumöhl, and Tomáš Vŷrost. 2021. ``YOLO Trading:
Riding the Herd During the Gamestop Episode.'' \emph{ZBW - Leibniz
Information Centre for Economics, Working Paper}.
\url{https://doi.org/http://hdl.handle.net/10419/230679}.

\leavevmode\hypertarget{ref-macey2021}{}%
Macey, Jonathan R. 2021. ``Securities Regulation as Class Warfare.''
\emph{Columbia Business Law Review, Forthcoming}, February.
\url{https://doi.org/http://dx.doi.org/10.2139/ssrn.3789706}.

\leavevmode\hypertarget{ref-dimuzio2021}{}%
Muzio, Tim Di. 2021. ``GameStop Capitalism. Wall Street Vs. The Reddit
Rally (Part 1).'' \emph{University of Wollongong Working Paper}, 1--13.
\url{https://doi.org/http://bnarchives.yorku.ca/673/}.

\leavevmode\hypertarget{ref-eaton2021}{}%
Pagano, Gregory W., Clifton T. Green, Brian Roseman, and Yanbin Wu.
2021. ``Zero-Commission Individual Investors, High Frequency Traders,
and Stock Market Quality.'' \emph{Available at SSRN:
Https://Ssrn.com/Abstract=3776874 or
Http://Dx.doi.org/10.2139/Ssrn.3776874}, January.
\url{https://doi.org/http://dx.doi.org/10.2139/ssrn.3776874}.

\leavevmode\hypertarget{ref-pagano2020}{}%
Pagano, Michael S., John Sedunov, and Raisa Velthuis. 2020. ``How Did
Retail Investors Respond to the Covid-19 Pandemic? The Effect of
Robinhood Brokerage Customers on Market Quality.'' \emph{Finance
Research Letters, Forthcoming}, November.
\href{https://doi.org/http://dx.doi.org/10.2139/ssrn.3703815\%20}{https://doi.org/http://dx.doi.org/10.2139/ssrn.3703815}.

\leavevmode\hypertarget{ref-peress2014}{}%
Peress, Joel. 2014. ``The Media and the Diffusion of Information in
Financial Markets: Evidence from Newspaper Strikes.'' \emph{The Journal
of Finance} 69 (5): 2007--43.
\url{https://doi.org/https://doi.org/10.1111/jofi.12179}.

\leavevmode\hypertarget{ref-tetlock2007}{}%
Tetlock, Paul C. 2007. ``Giving Content to Investor Sentiment: The Role
of Media in the Stock Market.'' \emph{The Journal of Finance} 62 (3):
1139--68.
\url{https://doi.org/https://doi.org/10.1111/j.1540-6261.2007.01232.x}.

\leavevmode\hypertarget{ref-tetlock2011}{}%
---------. 2011. ``All the News That's Fit to Reprint: Do Investors
React to Stale Information?'' \emph{The Review of Financial Studies} 24
(5): 1481--1512. \url{https://doi.org/10.1093/rfs/hhq141}.

\leavevmode\hypertarget{ref-wsbshop}{}%
Wallstreetbets News. 2020. ``Dissecting the Unique Lingo and Terminolgy
Used in the Subreddit R/Wallstreetbets.''
\url{\%22https://www.wallstreetbets.shop/blogs/news/dissecting-the-unique-lingo-and-terminology-used-in-the-subreddit-r-wallstreetbets\%22}.

\leavevmode\hypertarget{ref-welch2020}{}%
Welch, Ivo. 2020. ``The Wisdom of the Robinhood Crowd.'' \emph{NBER
Working Paper Series}, September.
\url{https://doi.org/http://www.nber.org/papers/w27866}.





\newpage
\singlespacing 
\end{document}
