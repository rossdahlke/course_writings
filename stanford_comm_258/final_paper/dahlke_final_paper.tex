\documentclass[12pt,]{article}
\usepackage[left=1in,top=1in,right=1in,bottom=1in]{geometry}
\newcommand*{\authorfont}{\fontfamily{phv}\selectfont}
\usepackage[]{mathpazo}


  \usepackage[T1]{fontenc}
  \usepackage[utf8]{inputenc}




\usepackage{abstract}
\renewcommand{\abstractname}{}    % clear the title
\renewcommand{\absnamepos}{empty} % originally center

\renewenvironment{abstract}
 {{%
    \setlength{\leftmargin}{0mm}
    \setlength{\rightmargin}{\leftmargin}%
  }%
  \relax}
 {\endlist}

\makeatletter
\def\@maketitle{%
  \newpage
%  \null
%  \vskip 2em%
%  \begin{center}%
  \let \footnote \thanks
    {\fontsize{18}{20}\selectfont\raggedright  \setlength{\parindent}{0pt} \@title \par}%
}
%\fi
\makeatother




\setcounter{secnumdepth}{0}

\usepackage{longtable,booktabs}



\title{Regime Change and Information Control  }



\author{\Large Ross Dahlke\vspace{0.05in} \newline\normalsize\emph{}  }


\date{}

\usepackage{titlesec}

\titleformat*{\section}{\normalsize\bfseries}
\titleformat*{\subsection}{\normalsize\itshape}
\titleformat*{\subsubsection}{\normalsize\itshape}
\titleformat*{\paragraph}{\normalsize\itshape}
\titleformat*{\subparagraph}{\normalsize\itshape}





\newtheorem{hypothesis}{Hypothesis}
\usepackage{setspace}


% set default figure placement to htbp
\makeatletter
\def\fps@figure{htbp}
\makeatother

\usepackage{graphicx}

% move the hyperref stuff down here, after header-includes, to allow for - \usepackage{hyperref}

\makeatletter
\@ifpackageloaded{hyperref}{}{%
\ifxetex
  \PassOptionsToPackage{hyphens}{url}\usepackage[setpagesize=false, % page size defined by xetex
              unicode=false, % unicode breaks when used with xetex
              xetex]{hyperref}
\else
  \PassOptionsToPackage{hyphens}{url}\usepackage[draft,unicode=true]{hyperref}
\fi
}

\@ifpackageloaded{color}{
    \PassOptionsToPackage{usenames,dvipsnames}{color}
}{%
    \usepackage[usenames,dvipsnames]{color}
}
\makeatother
\hypersetup{breaklinks=true,
            bookmarks=true,
            pdfauthor={Ross Dahlke ()},
             pdfkeywords = {},  
            pdftitle={Regime Change and Information Control},
            colorlinks=true,
            citecolor=blue,
            urlcolor=blue,
            linkcolor=magenta,
            pdfborder={0 0 0}}
\urlstyle{same}  % don't use monospace font for urls

% Add an option for endnotes. -----


% add tightlist ----------
\providecommand{\tightlist}{%
\setlength{\itemsep}{0pt}\setlength{\parskip}{0pt}}

% add some other packages ----------

% \usepackage{multicol}
% This should regulate where figures float
% See: https://tex.stackexchange.com/questions/2275/keeping-tables-figures-close-to-where-they-are-mentioned
\usepackage[section]{placeins}


\begin{document}
	
% \pagenumbering{arabic}% resets `page` counter to 1 
%
% \maketitle

{% \usefont{T1}{pnc}{m}{n}
\setlength{\parindent}{0pt}
\thispagestyle{plain}
{\fontsize{18}{20}\selectfont\raggedright 
\maketitle  % title \par  

}

{
   \vskip 13.5pt\relax \normalsize\fontsize{11}{12} 
\textbf{\authorfont Ross Dahlke} \hskip 15pt \emph{\small }   

}

}






\vskip -8.5pt


 % removetitleabstract

\noindent \doublespacing 

\begin{longtable}[]{@{}lll@{}}
\toprule
Method & koRpus & stringi\tabularnewline
\midrule
\endhead
Word count & 1885 & 1896\tabularnewline
Character count & 12827 & 12826\tabularnewline
Sentence count & 103 & Not available\tabularnewline
Reading time & 9.4 minutes & 9.5 minutes\tabularnewline
\bottomrule
\end{longtable}

Among comparative political scientists, the study of regimes around the
world centers on the question: What leads to countries adopting
democracy? In comparative politics literature, attempts at answering
this question primarily start by examining the relationship between
economic development and democratization. I start by examining the
economic argument and then move onto non-economic explanations.

This paper summarizes literature on a core subject of comparative
politics: regimes and regime change. In addition, I add a perspective on
how the literature relates to censorship, propaganda, and information
control. In total, the main goal of this final paper is to engage with a
fundamental topic in comparative politics and layer on top some ideas
from the field of mass communication.

\hypertarget{economic-development-and-democracy}{%
\section{Economic Development and
Democracy}\label{economic-development-and-democracy}}

There is a strong correlation between levels of economic development and
democracy in a country. However, there has been a historical divide on
the causality of economic development and the creation or maintenance of
democracy. Debates on the importance of economics in democratization was
distilled into two argument by Przeworski and Limongi (1997), the
``endogenous'' or ``exogenous'' models. In the endogenous view, economic
development spurs the emergence of democracy. In the exogenous model,
democracy emerges independent of economic development but is more likely
to survive in economically developed countries. In other words, economic
development either \emph{causes} democratization or \emph{maintains}
democratization caused by other processes.

Modernization is the focal point of the endogenous view of economic
development and democracy. Modernization of countries drives a sequence
of events of ``industrialization, urbanization, education,
communication, mobilization, and political incorporation'' (Przeworski
and Limongi 1997). Ultimately, modernization is an accumulation of
progressive social changes that readies a nation for its final
culmination: democracy. Under the endogenous theory of democratization,
there is a threshold of economic development at which we would expect
authoritarian regimes to transition to democracy.

Modernization also develops the communicative infrastructure and
processes that may be necessary for democracy. It is not just
modernization in the sense of technology and industrialization that
potentially induces democratization. Rather, the free flow of
information across networks may produce democracies. For example,
Castells (2010) refers to the output of modernization as the rise of
network society. As the world globalizes, the world is dominated by
central nodes which act as a connector for multiple types of network.
For example, New York City is a central hub for financial, news, and
entertainment networks. In order to modernize, a country must accept its
position within the networked world. Inevitably, this network position
opens up the country to more information flows which leads to
democratization.

In contrast, the exogenous view of democratization posits economic
development not cause of democracy but instead as a maintainer of
democracy. Lipset (1981) argues that the emergence of democracy is
independent of economic factors. However, the likelihood of a regime
surviving increases in more developed nations. Instead of the threshold
of democratization offered by the endogenous view, there is a threshold
of stability for regimes. There is a ``U'' shape that develops on the
likelihood of a dictatorship transitioning to democracy where
authoritarian regimes with low and high levels modernization are likely
to persist. This exogenous view has withood empirical scrutiny whereas
the endogenous view has not.

Put together, there is little causal explanatory power of economic
development bringing down authoritarian regimes and transitioning them
to democracy. Recent examples such as China, Singapore and Rhwanda have
expriernced dramatic recent economic growth but remain authoritarian
states at various levels. Instead, once established, democracies that
are wealthier and more modernized are more likely to survive. Further,
Lipset theorized that democracies are more likely to be destabilized
when their economies grow quickly. These accelerative periods breed
``extremist moments.'' For example, fascism and communism were products
of rapid economic development. However, the evidence suggests that
instead of authoritarianism growing during economic advancement that
democracies fall during economic contractions (Przeworski and Limongi
1997).

In summary, economic development is crucial towards maintaining
democracy but does not itself cause democratic transition. There are
other factors that cause regime change. Instead, the level of
modernization that predicts whether the new government will be
maintained. Therefore, the debate among endogenous and exogenous view of
economic and democracy is closed and now the field examines other causes
of democratization.

\hypertarget{factors-of-regime-change}{%
\section{Factors of Regime Change}\label{factors-of-regime-change}}

If economic development does not cause regime change, what does?
Robinson and Acemoglu (2006) place the struggle between elites and
citizens as central to the development of democracy. Rustow (1971)
summarizes a number of explanations for democratization including
features of the conflict and reconciliation of groups in a society that
form social and political structure and the need for certain beliefs and
psychological attitudes among the citizenry.

While it is not economic development \emph{per say} that causes
nondemocracies to transition to democracy, one possibility is that
social struggle, largely economic, between elites and citizens drives
democratization (Robinson and Acemoglu 2006). In this view, these two
groups have well-defined economic preferences. Elites in a country want
to maintain authoritarian rule because it provides them with higher
levels of economic wealth. Whereas the majority of citizens want
democratic institutions under the assumption that they benefit from them
and therefore will exert their power in an effort to attain democracy.
The dueling preferences of these two groups are resolved through an
inherently conflictual political process where de jure and de facto
political power is wielded against one another. The balance of power
between elites and citizens determines whether a society transitions
from nondemocracy to democracy and perhaps if it reverts back to
nondemocracy later.

There are factors that can alter the levels of political power
controlled by citizens and elites (Robinson and Acemoglu 2006). Civil
society and the effective threat of revolution, particularly the level
of self-organization of citizens, can shift the balance of power towards
democracy. Exogenous shocks and crises in a country can create a
transition of de facto power to citizens and make revolution less costly
to carry out. The sources of income and composition of wealth, such as
elites being heavily invested in land versus physical and human capital,
can create power imbalances that make democratization more likely.
Political institutions can preemptively use repression to avoid
democratization. Inter-group inequality, for example if a citizenry is
divided along racial or religious lines, can make democratization
more-or-less likely. A large middle class can act as a buffer of
conflict between elites and the citizens. Globalization can shift elite
exposure to international financial markets. And political identities
can create citizens with varying levels of solidarity. All of these
factors impact the balance of de facto and de jure power that can be
used by citizens or elites. This balance of power, according to Robinson
and Acemoglu (2006), is what ultimately creates transitions to
democracy.

In addition, information flows can also alter the levels of political
power between citizens and elites. Free flowing information can help
citizens organize politically to increase their de facto power.
Authoritarian regimes participate in various types of censorship in
order to quell citizens' de facto power. In addition, regimes can
participate in propaganda in order to increase the legitimacy of their
government in order to communicate its de jure power. For example, a
regime can participate in propaganda as signalling in order to maintain
its perceived level of control across the nation and also to show that
it has the power in the first place to enforce compliance with the
regime's wishes.

Rustow (1971) offers potential explanations under which democracy can
thrive. One of these explanations is similar to the democratization
thesis from Robinson (2006) in that it insists that conflict and
reconciliation of conflicts are essential components to democracy. For
example, there is a vitality of American institutions because of
citizens' ``multiple memberships in potential groups.'' These
crosscutting associations mediate American politics and allow different
coalitions to emerge within the democratic process--a process that is
self-reinforcing. In addition, Rustow points to arguments that
democratic stability requires a commitment to democratic values among
professional politicians. These democratic values are created and
maintained through politicians' links to one another through ties of
political organizations.

McFaul's (2002) analysis of the fourth wave of democracy, or
democratization in a postcommunist world, also hinges on the tenant of
the balance of power dictating transitions to and from democracy.
Decommunization triggered this fourth wave of regime change, but unlike
prior waves, the fourth wave produced regime change both to democracy
\emph{and} dictatorship. The ideological orientation of the more
powerful party that enjoyed an asymmetrical balance of power in a
country largely determined the type of regime that emerged out of the
fourth wave. Therefore, democracy did emerge in some countries, but it
was only where pro-democracy forces enjoyed a decisive power advantage.

Another potential explanation put forth by Rustow is that there is a
need for certain beliefs of psychological attitudes among citizens. In
other words, there is a certain culture of democracy that is needed
among the masses in order to maintain a democracy. There either needs to
be a consensus in common beliefs and values or in the procedure to the
rules of the game. There is an ``agreement to differ.'' There are
potentially deeper psychological and civic attitudes such as empathy and
a willingness to participate that underline this culture of democracy.
However, it is not just participation, but also ``traditional or
parochial'' attitudes that are the bedrock of a ``civic culture'' that
is necessary for democracy. Propaganda can help to shape citizens'
psychological attitudes in relation to the values necessary for
democracy. A regime can participate in persuasion as signaling to create
or reinforce nondemocratic attitudes.

Huntington's (1991) analysis of Democracy's Third Wave also points to
culture as being one of the obstacles to democratization. Cultural
traditions set the attitudes, values, beliefs, and related behavior
patterns that determine whether a society is conducive to democracy or
is antidemocratic. Specifically, Huntington points to Confucianism and
Islam as two cultures which are inherently antidemocratic. He says that
Confucianism is a culture that emphasizes the collective over the
individual and there is a lack of traditional rights against the state.
Unlike America, which has a notion of ``God-given'' rights which are
unalienable, Confucianism's rights are created by the state. Huntington
also points to Islam as an inherently antidemocratic culture because of
the lack of separation between church and state. In Islam, the leader of
the church is also the leader of the state. Therefore, it is only a
divine power, not the consent of the governed, that allows for a leader.
Both of these examples feel to me to be too inductive. It is easy to
look back on history and create strings that tell a story of the
present. The same exercise could be applied to post-World War II Germany
and conclude that German culture is not conducive to democracy. In the
present, the argument that Germany cannot sustain democracy because of
its culture seems unfounded.

\hypertarget{conclusion}{%
\section{Conclusion}\label{conclusion}}

Historically, the debate around democratization in comparative politics
has centered around economic development. Economic development may not
cause democratization, but development can maintain democracies that are
established for other reasons. Some of the reasons that democracy can be
established are the power balance between citizens and elites,
psychological attitudes and beliefs towards democracy, and other
cultural factors. Further, the flow of information, including
information control via propagnda and censorship, plays a crucial role
in these democratizing factors.

\hypertarget{references}{%
\section*{References}\label{references}}
\addcontentsline{toc}{section}{References}

\hypertarget{refs}{}
\leavevmode\hypertarget{ref-castells}{}%
Castells, Manuel. 2010. \emph{The Rise of the Network Society}. 2nd ed.
Vol. 1. The Information Age. Chichester: Wiley-Blackwell.

\leavevmode\hypertarget{ref-huntington}{}%
Huntington, Samuel P. 1991. ``Democracy's Third Wave.'' \emph{Journal of
Democracy}, 12--34.

\leavevmode\hypertarget{ref-lipset}{}%
Lipset, S. M., and Johns Hopkins University Press. 1981. \emph{Political
Man: The Social Bases of Politics}. A Johns Hopkins Paperback. Johns
Hopkins University Press.
\url{https://books.google.com/books?id=Gv-CAAAAMAAJ}.

\leavevmode\hypertarget{ref-mcfaul}{}%
McFaul, Michael. 2002. ``The Fourth Wave of Democracy and Dictatorship:
Noncooperative Transitions in the Postcommunist World.'' \emph{World
Politics} 54 (2): 212--44. \url{http://www.jstor.org/stable/25054183}.

\leavevmode\hypertarget{ref-przeworski}{}%
Przeworski, Adam, and Fernando Limongi. 1997. ``Modernization: Theories
and Facts.'' \emph{World Politics} 49 (2): 155--83.
\url{https://doi.org/10.1353/wp.1997.0004}.

\leavevmode\hypertarget{ref-robinson}{}%
Robinson, James A., and Daron Acemoglu. 2006. \emph{Economic Origins of
Dictatorship and Democracy}. Cambridge, UK: Cambridge University Press;
Cambridge University Press.
\url{http://www.cambridge.org/us/catalogue/catalogue.asp?isbn=0521855268}.

\leavevmode\hypertarget{ref-rustow}{}%
Rustow, Dankwart A. 1971. ``Agreement, Dissent, and Democratic
Fundamentals.'' In \emph{Theory and Politics/Theorie Und Politik:
Festschrift Zum 70. Geburtstag Für Carl Joachim Friedrich}, edited by
Klaus Von Beyme, 328--42. Dordrecht: Springer Netherlands.
\url{https://doi.org/10.1007/978-94-010-2750-2_17}.





\newpage
\singlespacing 
\end{document}
