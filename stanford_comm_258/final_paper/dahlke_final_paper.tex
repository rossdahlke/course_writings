\documentclass[12pt,]{article}
\usepackage[left=1in,top=1in,right=1in,bottom=1in]{geometry}
\newcommand*{\authorfont}{\fontfamily{phv}\selectfont}
\usepackage[]{mathpazo}


  \usepackage[T1]{fontenc}
  \usepackage[utf8]{inputenc}




\usepackage{abstract}
\renewcommand{\abstractname}{}    % clear the title
\renewcommand{\absnamepos}{empty} % originally center

\renewenvironment{abstract}
 {{%
    \setlength{\leftmargin}{0mm}
    \setlength{\rightmargin}{\leftmargin}%
  }%
  \relax}
 {\endlist}

\makeatletter
\def\@maketitle{%
  \newpage
%  \null
%  \vskip 2em%
%  \begin{center}%
  \let \footnote \thanks
    {\fontsize{18}{20}\selectfont\raggedright  \setlength{\parindent}{0pt} \@title \par}%
}
%\fi
\makeatother




\setcounter{secnumdepth}{0}

\usepackage{longtable,booktabs}



\title{Regime Change and Information Control  }



\author{\Large Ross Dahlke\vspace{0.05in} \newline\normalsize\emph{}  }


\date{}

\usepackage{titlesec}

\titleformat*{\section}{\normalsize\bfseries}
\titleformat*{\subsection}{\normalsize\itshape}
\titleformat*{\subsubsection}{\normalsize\itshape}
\titleformat*{\paragraph}{\normalsize\itshape}
\titleformat*{\subparagraph}{\normalsize\itshape}





\newtheorem{hypothesis}{Hypothesis}
\usepackage{setspace}


% set default figure placement to htbp
\makeatletter
\def\fps@figure{htbp}
\makeatother

\usepackage{graphicx}

% move the hyperref stuff down here, after header-includes, to allow for - \usepackage{hyperref}

\makeatletter
\@ifpackageloaded{hyperref}{}{%
\ifxetex
  \PassOptionsToPackage{hyphens}{url}\usepackage[setpagesize=false, % page size defined by xetex
              unicode=false, % unicode breaks when used with xetex
              xetex]{hyperref}
\else
  \PassOptionsToPackage{hyphens}{url}\usepackage[draft,unicode=true]{hyperref}
\fi
}

\@ifpackageloaded{color}{
    \PassOptionsToPackage{usenames,dvipsnames}{color}
}{%
    \usepackage[usenames,dvipsnames]{color}
}
\makeatother
\hypersetup{breaklinks=true,
            bookmarks=true,
            pdfauthor={Ross Dahlke ()},
             pdfkeywords = {},  
            pdftitle={Regime Change and Information Control},
            colorlinks=true,
            citecolor=blue,
            urlcolor=blue,
            linkcolor=magenta,
            pdfborder={0 0 0}}
\urlstyle{same}  % don't use monospace font for urls

% Add an option for endnotes. -----


% add tightlist ----------
\providecommand{\tightlist}{%
\setlength{\itemsep}{0pt}\setlength{\parskip}{0pt}}

% add some other packages ----------

% \usepackage{multicol}
% This should regulate where figures float
% See: https://tex.stackexchange.com/questions/2275/keeping-tables-figures-close-to-where-they-are-mentioned
\usepackage[section]{placeins}


\begin{document}
	
% \pagenumbering{arabic}% resets `page` counter to 1 
%
% \maketitle

{% \usefont{T1}{pnc}{m}{n}
\setlength{\parindent}{0pt}
\thispagestyle{plain}
{\fontsize{18}{20}\selectfont\raggedright 
\maketitle  % title \par  

}

{
   \vskip 13.5pt\relax \normalsize\fontsize{11}{12} 
\textbf{\authorfont Ross Dahlke} \hskip 15pt \emph{\small }   

}

}








\begin{abstract}

    \hbox{\vrule height .2pt width 39.14pc}

    \vskip 8.5pt % \small 

\noindent This paper summarizes literature on a core subject of comparative
politics: regimes and regime change. In addition, I add a perspective on
how the literature relates to censorship, propaganda, and information
control. In total, the main goal of this final paper is to give me an
opportunity to learn more about the fundamentals of comparative politics
and layer on top some ideas from the field of mass communication.


    \hbox{\vrule height .2pt width 39.14pc}


\end{abstract}


\vskip -8.5pt


 % removetitleabstract

\noindent \doublespacing 

\begin{longtable}[]{@{}lll@{}}
\toprule
Method & koRpus & stringi\tabularnewline
\midrule
\endhead
Word count & 1462 & 1472\tabularnewline
Character count & 9873 & 9872\tabularnewline
Sentence count & 79 & Not available\tabularnewline
Reading time & 7.3 minutes & 7.4 minutes\tabularnewline
\bottomrule
\end{longtable}

Democracy is a founding principle in the United States and has persisted
through its history. This collective value has not only meant that
American have taken their own democracy for granted, it also leads to an
implicit worldview that centers democracy as the ultimate goal for
countries around the world. As a result, political scientists have
historically framed the study of regimes around the world around the
question: What leads to countries adopting democracy? {[}Need coherent
thesis{]}

\hypertarget{economic-development-and-democracy}{%
\section{Economic Development and
Democracy}\label{economic-development-and-democracy}}

There is a strong correlation between levels of economic development and
democracy in a country. However, there has been a historical divide on
the causality of economic development and the creation or maintenance of
democracy. Przeworski and Limogni (1997) distill two arguments, either
economic development is ``endogenous'' and spurs the emergence of
democracy; or it is ``exogenous,'' and democracy emerges independent of
economic development but is more likely to survive in economically
developed countries. In other words, economic development either causes
democratization or maintains democratization caused by other processes.

The endogenous view of economic development and democracy is one of
modernization. Modernization of countries drives a sequences of events
of ``industrialization, urbanization, education, communication,
mobilization, and political incorporation'' (Przeworski and Limongi
1997). Ultimately, modernization is an accumulation of progressive
social changes that readies a nation for it's final culmination:
democracy. Under the endogenous theory of democratization, there is a
threshold of economic development at which we would expect authoritarian
regimes to transition to democracy. However, this theory does not hold
up under empirical testing. Instead, evidence points towards the
exogenous view of economic development and democracy.

The exogenous view of democratization posits economic development not as
the driver towards democracy but instead as a maintainer of democracy.
Lipset (1981) argues that the emergence of democracy is independent of
economic factors. However, the likelihood of a regime surviving
increases in more developed nations. Instead of the threshold of
democratization offered by the endogenous view, there is a threshold of
stability for regimes. There is a ``U'' shape that develops on the
likelihood of a dictatorship transitioning to democracy where
authoritarian regimes with low and high levels modernization are likely
to persist.

Put together, there is little causal explanatory power of economic
development bringing down authoritarian regimes and transitioning them
to democracy. Instead, once established, democracies that are wealthier
and more modernized are more likely to survive. Further, Lipset even
went so far as to theorize that democracies are more likely to be
destabilized when their economies grow quickly. These accelerative
periods breed ``extremist moments.'' For example, fascism and communism
were produts of rapid economic development. However, it appears more
likely that instead of authoritarianism growing during economic
advancement that democracies fall during economic contractions
(Przeworski and Limongi 1997).

In summary, economic development does not breed democracies. In
contrast, there are other factors that cause regime change. It is the
level of modernization that predicts whether the new government will be
maintained. In addition, economic crises can lead to democracies falling
into authoritarian states.

\hypertarget{factors-of-regime-change}{%
\section{Factors of Regime Change}\label{factors-of-regime-change}}

If economic development does not cause regime change, what does?
Robinson and Acemoglu (2006) place the struggle between elites and
citizens as central to the development of democracy. Rustow (1971)
summarizes a number of explanations for democratization including
features of the conflict and reconciliation of groups in a society that
form social and political structure and the need for certain beliefs and
psychological attitudes among the citizenry.

While it is not economic development \emph{per say} that causes
nondemocracies to transition to democracy, one possibility is that
social struggle, largely economic, between elites and citizens drives
democratization (Robinson and Acemoglu 2006). In this view, these two
groups have well-defined economic preferences. Elites in a country want
to maintain authoritarian rule because it provides them with higher
levels of economic wealth. Whereas the majority of citizens want
democratic institutions under the assumption that they benefit from them
and therefore will exert their power in an effort to attain democracy.
The dueling preferences of these two groups are resolved through an
inherently conflictual political process where de jure and de facto
political power is wielded against one another. The balance of power
between elites and citizens determines whether a society transitions
from nondemocracy to democracy and perhaps if it reverts back to
nondemocracy later.

There are factors that can alter the levels of political power
controlled by citizens and elites (Robinson and Acemoglu 2006). Civil
society and the effective threat of revolution, particularly the level
of self-organization of citizens, can shift the balance of power towards
democracy. Exogenous shocks and crises in a country can create a
transition of de facto power to citizens and make revolution less costly
to carry out. The sources of income and composition of wealth, such as
elites being heavily invested in land versus physical and human capital,
can create power imbalances that make democratization more likely.
Political institutions can preemptively use repression to avoid
democratization. Inter-group inequality, for example if a citizenry is
divided along racial or religious lines, can make democratization
more-or-less likely. A large middle class can act as a buffer of
conflict between elites and the citizens. Globalization can shift elite
exposure to international financial markets. And political identities
can create citizens with varying levels of solidarity. All of these
factors impact the balance of de facto and de jure power that can be
used by citizens or elites. This balance of power, according to Robinson
and Acemoglu (2006), is what ultimately create transitions to democracy.

Rustow (1971) offers potential explanations under which democracy can
thrive. One of these explanations is similar to the democratization
thesis from Robinsin (2006) in that it insists that conflict and
reconciliation of conflicts are essential components to democracy. For
example, there is a vitality of American institutions because of
citizens' ``multiple memberships in potential groups.'' These
crosscutting associations mediate American politics and allow different
coalitions to emerge within the democratic process--a process that is
self-reinforcing. In addition, Rustow points to arguments that
democratic stability requires a commitment to democratic values among
professional politicians. These democratic values are created and
maintained through politicians' links to one another through ties of
political organizations.

McFaul's (2002) analysis of the fourth wave of democracy, or
democratization in a postcommunist world, also hinges on tenant of the
balance of power dictating transitions to and from democracy.
Decommunization triggered this fourth wave of regime change, but unlike
prior waves, the fourth wave produce regime change both to democracy
\emph{and} dictatorship. The ideological orientation of the more
powerful party that enjoyed an asymmetrical balance of power in a
country largely determined the type of regime that emerged out of the
fourth wave. Therefore, democracy did emerge in some countries, but it
was only where pro-democracy forces enjoyed a decisive power advantage.

Another potential explanation put forth by Rustow is that there is a
need for certain beliefs of psychological attitudes among citizens. In
other words, there is a certain culture of democracy that is needed
among the masses in order to maintain a democracy. There either needs to
be a consensus in common beliefs and values or in the procedure to the
rules of the game. There is an ``agreement to differ.'' There are
potentially deeper psychological and civic attitudes such as empathy and
a willingness to participate that underline this culture of democracy.
However, it just not participate, but also ``traditional or parochial''
attitudes that are the bedrock of a ``civic culture'' that is necessary
for democracy.

Huntington's (1991) analysis of Democracy's Third Wave also points to
culture as being one of the obstacles to democratization. Cultural
traditions set the attitudes, values, beliefs, and related behavior
patterns that determine whether a society is conducive to democracy or
is antidemocratic. Specifically, Huntington points to Confucianism and
Islam as two cultures which are inherently antidemocratic. He says that
Confucianism is a culture that emphasizes the collective over the
individual and there is a lack of traditional rights against the state.
Unlike America, which has a notion of ``God-given'' rights which are
unalienable, Confucianism's rights are created by the state. Huntington
also points to Islam as an inherently antidemocratic culture because of
the lack of separation between church and state. In Islam, the leader of
the church is also the leader of the state. Therefore, it is only a
divine power, not the consent of the governed, that allows for a leader.
Both of these examples feel to me to be too inductive. It is easy to
look back on history and create strings that tell a story of the
present. The same exercise could be applied to post-World War II Germany
and conclude that German culture is not conducive to democracy. In the
present, that argument that Germany cannot sustain democracy because of
its culture seems unfounded.

\hypertarget{references}{%
\section*{References}\label{references}}
\addcontentsline{toc}{section}{References}

\hypertarget{refs}{}
\leavevmode\hypertarget{ref-huntington}{}%
Huntington, Samuel P. 1991. ``Democracy's Third Wave.'' \emph{Journal of
Democracy}, 12--34.

\leavevmode\hypertarget{ref-lipset}{}%
Lipset, S. M., and Johns Hopkins University Press. 1981. \emph{Political
Man: The Social Bases of Politics}. A Johns Hopkins Paperback. Johns
Hopkins University Press.
\url{https://books.google.com/books?id=Gv-CAAAAMAAJ}.

\leavevmode\hypertarget{ref-mcfaul}{}%
McFaul, Michael. 2002. ``The Fourth Wave of Democracy and Dictatorship:
Noncooperative Transitions in the Postcommunist World.'' \emph{World
Politics} 54 (2): 212--44. \url{http://www.jstor.org/stable/25054183}.

\leavevmode\hypertarget{ref-przeworski}{}%
Przeworski, Adam, and Fernando Limongi. 1997. ``Modernization: Theories
and Facts.'' \emph{World Politics} 49 (2): 155--83.
\url{https://doi.org/10.1353/wp.1997.0004}.

\leavevmode\hypertarget{ref-robinson}{}%
Robinson, James A., and Daron Acemoglu. 2006. \emph{Economic Origins of
Dictatorship and Democracy}. Cambridge, UK: Cambridge University Press;
Cambridge University Press.
\url{http://www.cambridge.org/us/catalogue/catalogue.asp?isbn=0521855268}.

\leavevmode\hypertarget{ref-rustow}{}%
Rustow, Dankwart A. 1971. ``Agreement, Dissent, and Democratic
Fundamentals.'' In \emph{Theory and Politics/Theorie Und Politik:
Festschrift Zum 70. Geburtstag Für Carl Joachim Friedrich}, edited by
Klaus Von Beyme, 328--42. Dordrecht: Springer Netherlands.
\url{https://doi.org/10.1007/978-94-010-2750-2_17}.





\newpage
\singlespacing 
\end{document}
